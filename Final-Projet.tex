% Options for packages loaded elsewhere
\PassOptionsToPackage{unicode}{hyperref}
\PassOptionsToPackage{hyphens}{url}
%
\documentclass[
]{article}
\usepackage{amsmath,amssymb}
\usepackage{iftex}
\ifPDFTeX
  \usepackage[T1]{fontenc}
  \usepackage[utf8]{inputenc}
  \usepackage{textcomp} % provide euro and other symbols
\else % if luatex or xetex
  \usepackage{unicode-math} % this also loads fontspec
  \defaultfontfeatures{Scale=MatchLowercase}
  \defaultfontfeatures[\rmfamily]{Ligatures=TeX,Scale=1}
\fi
\usepackage{lmodern}
\ifPDFTeX\else
  % xetex/luatex font selection
\fi
% Use upquote if available, for straight quotes in verbatim environments
\IfFileExists{upquote.sty}{\usepackage{upquote}}{}
\IfFileExists{microtype.sty}{% use microtype if available
  \usepackage[]{microtype}
  \UseMicrotypeSet[protrusion]{basicmath} % disable protrusion for tt fonts
}{}
\makeatletter
\@ifundefined{KOMAClassName}{% if non-KOMA class
  \IfFileExists{parskip.sty}{%
    \usepackage{parskip}
  }{% else
    \setlength{\parindent}{0pt}
    \setlength{\parskip}{6pt plus 2pt minus 1pt}}
}{% if KOMA class
  \KOMAoptions{parskip=half}}
\makeatother
\usepackage{xcolor}
\usepackage[margin=1in]{geometry}
\usepackage{color}
\usepackage{fancyvrb}
\newcommand{\VerbBar}{|}
\newcommand{\VERB}{\Verb[commandchars=\\\{\}]}
\DefineVerbatimEnvironment{Highlighting}{Verbatim}{commandchars=\\\{\}}
% Add ',fontsize=\small' for more characters per line
\usepackage{framed}
\definecolor{shadecolor}{RGB}{248,248,248}
\newenvironment{Shaded}{\begin{snugshade}}{\end{snugshade}}
\newcommand{\AlertTok}[1]{\textcolor[rgb]{0.94,0.16,0.16}{#1}}
\newcommand{\AnnotationTok}[1]{\textcolor[rgb]{0.56,0.35,0.01}{\textbf{\textit{#1}}}}
\newcommand{\AttributeTok}[1]{\textcolor[rgb]{0.13,0.29,0.53}{#1}}
\newcommand{\BaseNTok}[1]{\textcolor[rgb]{0.00,0.00,0.81}{#1}}
\newcommand{\BuiltInTok}[1]{#1}
\newcommand{\CharTok}[1]{\textcolor[rgb]{0.31,0.60,0.02}{#1}}
\newcommand{\CommentTok}[1]{\textcolor[rgb]{0.56,0.35,0.01}{\textit{#1}}}
\newcommand{\CommentVarTok}[1]{\textcolor[rgb]{0.56,0.35,0.01}{\textbf{\textit{#1}}}}
\newcommand{\ConstantTok}[1]{\textcolor[rgb]{0.56,0.35,0.01}{#1}}
\newcommand{\ControlFlowTok}[1]{\textcolor[rgb]{0.13,0.29,0.53}{\textbf{#1}}}
\newcommand{\DataTypeTok}[1]{\textcolor[rgb]{0.13,0.29,0.53}{#1}}
\newcommand{\DecValTok}[1]{\textcolor[rgb]{0.00,0.00,0.81}{#1}}
\newcommand{\DocumentationTok}[1]{\textcolor[rgb]{0.56,0.35,0.01}{\textbf{\textit{#1}}}}
\newcommand{\ErrorTok}[1]{\textcolor[rgb]{0.64,0.00,0.00}{\textbf{#1}}}
\newcommand{\ExtensionTok}[1]{#1}
\newcommand{\FloatTok}[1]{\textcolor[rgb]{0.00,0.00,0.81}{#1}}
\newcommand{\FunctionTok}[1]{\textcolor[rgb]{0.13,0.29,0.53}{\textbf{#1}}}
\newcommand{\ImportTok}[1]{#1}
\newcommand{\InformationTok}[1]{\textcolor[rgb]{0.56,0.35,0.01}{\textbf{\textit{#1}}}}
\newcommand{\KeywordTok}[1]{\textcolor[rgb]{0.13,0.29,0.53}{\textbf{#1}}}
\newcommand{\NormalTok}[1]{#1}
\newcommand{\OperatorTok}[1]{\textcolor[rgb]{0.81,0.36,0.00}{\textbf{#1}}}
\newcommand{\OtherTok}[1]{\textcolor[rgb]{0.56,0.35,0.01}{#1}}
\newcommand{\PreprocessorTok}[1]{\textcolor[rgb]{0.56,0.35,0.01}{\textit{#1}}}
\newcommand{\RegionMarkerTok}[1]{#1}
\newcommand{\SpecialCharTok}[1]{\textcolor[rgb]{0.81,0.36,0.00}{\textbf{#1}}}
\newcommand{\SpecialStringTok}[1]{\textcolor[rgb]{0.31,0.60,0.02}{#1}}
\newcommand{\StringTok}[1]{\textcolor[rgb]{0.31,0.60,0.02}{#1}}
\newcommand{\VariableTok}[1]{\textcolor[rgb]{0.00,0.00,0.00}{#1}}
\newcommand{\VerbatimStringTok}[1]{\textcolor[rgb]{0.31,0.60,0.02}{#1}}
\newcommand{\WarningTok}[1]{\textcolor[rgb]{0.56,0.35,0.01}{\textbf{\textit{#1}}}}
\usepackage{longtable,booktabs,array}
\usepackage{calc} % for calculating minipage widths
% Correct order of tables after \paragraph or \subparagraph
\usepackage{etoolbox}
\makeatletter
\patchcmd\longtable{\par}{\if@noskipsec\mbox{}\fi\par}{}{}
\makeatother
% Allow footnotes in longtable head/foot
\IfFileExists{footnotehyper.sty}{\usepackage{footnotehyper}}{\usepackage{footnote}}
\makesavenoteenv{longtable}
\usepackage{graphicx}
\makeatletter
\def\maxwidth{\ifdim\Gin@nat@width>\linewidth\linewidth\else\Gin@nat@width\fi}
\def\maxheight{\ifdim\Gin@nat@height>\textheight\textheight\else\Gin@nat@height\fi}
\makeatother
% Scale images if necessary, so that they will not overflow the page
% margins by default, and it is still possible to overwrite the defaults
% using explicit options in \includegraphics[width, height, ...]{}
\setkeys{Gin}{width=\maxwidth,height=\maxheight,keepaspectratio}
% Set default figure placement to htbp
\makeatletter
\def\fps@figure{htbp}
\makeatother
\setlength{\emergencystretch}{3em} % prevent overfull lines
\providecommand{\tightlist}{%
  \setlength{\itemsep}{0pt}\setlength{\parskip}{0pt}}
\setcounter{secnumdepth}{-\maxdimen} % remove section numbering
\usepackage{booktabs}
\usepackage{longtable}
\usepackage{array}
\usepackage{multirow}
\usepackage{wrapfig}
\usepackage{float}
\usepackage{colortbl}
\usepackage{pdflscape}
\usepackage{tabu}
\usepackage{threeparttable}
\usepackage{threeparttablex}
\usepackage[normalem]{ulem}
\usepackage{makecell}
\usepackage{xcolor}
\ifLuaTeX
  \usepackage{selnolig}  % disable illegal ligatures
\fi
\usepackage{bookmark}
\IfFileExists{xurl.sty}{\usepackage{xurl}}{} % add URL line breaks if available
\urlstyle{same}
\hypersetup{
  pdftitle={MSDS 607 --- Final Project: Predicting Onset of Diabetes (Pima Indians Dataset)},
  pdfauthor={Taha Malik (malmal6565)},
  hidelinks,
  pdfcreator={LaTeX via pandoc}}

\title{MSDS 607 --- Final Project: Predicting Onset of Diabetes (Pima
Indians Dataset)}
\author{Taha Malik (malmal6565)}
\date{2025-12-17}

\begin{document}
\maketitle

{
\setcounter{tocdepth}{3}
\tableofcontents
}
\subsection{Setup and package
installation}\label{setup-and-package-installation}

Description: This chunk loads required libraries and optionally installs
missing packages. I keep \texttt{install.packages()} commented so you
can uncomment and run once if needed. The rest of the code assumes these
libraries are available.

Expected output: - No printed output if libraries load successfully;
otherwise R will show messages about missing packages.

\begin{Shaded}
\begin{Highlighting}[]
\CommentTok{\# Uncomment the following lines if you need to install packages:}
\CommentTok{\# packages \textless{}{-} c("tidyverse","caret","pROC","randomForest","xgboost","vip","fastshap",}
\CommentTok{\#               "patchwork","gridExtra","DALEX","mice","corrplot","DT","kableExtra",}
\CommentTok{\#               "skimr","rmarkdown","OpenML")}
\CommentTok{\# install.packages(setdiff(packages, rownames(installed.packages())))}
\FunctionTok{options}\NormalTok{(}\AttributeTok{skimr.inline =} \ConstantTok{FALSE}\NormalTok{)}
\CommentTok{\# Load libraries}
\FunctionTok{library}\NormalTok{(tidyverse)}
\FunctionTok{library}\NormalTok{(caret)}
\FunctionTok{library}\NormalTok{(pROC)}
\FunctionTok{library}\NormalTok{(randomForest)}
\FunctionTok{library}\NormalTok{(xgboost)}
\FunctionTok{library}\NormalTok{(vip)       }\CommentTok{\# variable importance visualization}
\FunctionTok{library}\NormalTok{(fastshap)  }\CommentTok{\# SHAP explanations}
\CommentTok{\#library(patchwork)}
\FunctionTok{library}\NormalTok{(gridExtra)}
\FunctionTok{library}\NormalTok{(mice)      }\CommentTok{\# multiple imputation (if used)}
\FunctionTok{library}\NormalTok{(corrplot)}
\FunctionTok{library}\NormalTok{(skimr)}
\FunctionTok{library}\NormalTok{(knitr)}
\FunctionTok{library}\NormalTok{(kableExtra)}
\CommentTok{\# OpenML client for dataset metadata (if internet available)}
\CommentTok{\# library(OpenML)}

\FunctionTok{set.seed}\NormalTok{(}\DecValTok{123}\NormalTok{)      }\CommentTok{\# reproducible randomness}
\FunctionTok{options}\NormalTok{(}\AttributeTok{scipen =} \DecValTok{999}\NormalTok{)}
\end{Highlighting}
\end{Shaded}

\subsection{Abstract}\label{abstract}

One paragraph summary of the project and its deliverables.

I build predictive models for diabetes onset using the Pima Indians
dataset (provided CSV) and supporting dataset metadata fetched from
OpenML. The work follows the OSEMN workflow: obtain (CSV + OpenML
metadata), scrub (handle biologically implausible zeros and
missingness), explore (descriptive statistics and visualizations), model
(logistic regression baseline, random forest, and XGBoost with
cross-validation and hyperparameter tuning), and interpret (ROC/AUC,
calibration curves, variable importance and SHAP explanations).
Deliverables include a reproducible RMarkdown report, a GitHub
repository with code and data, and a slide deck generated from R
Markdown. The final report maps exactly to the MSDS 607 checklist items
to maximize rubric points.

\subsection{Literature review \& clinical
context}\label{literature-review-clinical-context}

Description: Short literature context to motivate the analysis.
Citations are provided for classical sources and domain context. You
should include these references in your presentation slide notes.

Key references (short): - Smith et al.~(1988) --- original dataset paper
describing the ADAP algorithm on the Pima Indians dataset. - American
Diabetes Association (ADA) and WHO --- background on diabetes prevalence
and importance of early detection (cite in presentation). - Recent ML in
medicine overviews (e.g., Ribeiro et al., Lundberg \& Lee) for
interpretability (SHAP/LIME).

Expected output: - Narrative text for the report; no code output.

\subsection{Data and provenance}\label{data-and-provenance}

Description: I use two types of data sources to satisfy the rubric: 1.
Local CSV: \texttt{diabetes.csv} (individual-level measurements). 2.
Web/API metadata: OpenML dataset (dataset id 37) for provenance and
variable description.

The code below reads the CSV and (optionally) fetches OpenML metadata.
If you are offline, the OpenML chunk is marked eval=FALSE so it will not
run. The CSV must be present in the repo.

Expected output: - Glimpse and summary of the dataset (rows/columns). -
Table of Outcome counts.

\begin{Shaded}
\begin{Highlighting}[]
\CommentTok{\# Read local CSV (ensure file is in working directory or in data/ directory)}
\NormalTok{csv\_path }\OtherTok{\textless{}{-}} \StringTok{"diabetes.csv"}
\ControlFlowTok{if}\NormalTok{(}\SpecialCharTok{!}\FunctionTok{file.exists}\NormalTok{(csv\_path)) \{}
  \FunctionTok{stop}\NormalTok{(}\StringTok{"Please make sure \textquotesingle{}diabetes.csv\textquotesingle{} is in the working directory."}\NormalTok{)}
\NormalTok{\}}
\NormalTok{diabetes }\OtherTok{\textless{}{-}}\NormalTok{ readr}\SpecialCharTok{::}\FunctionTok{read\_csv}\NormalTok{(csv\_path, }\AttributeTok{show\_col\_types =} \ConstantTok{FALSE}\NormalTok{)}

\CommentTok{\# Basic structure}
\FunctionTok{glimpse}\NormalTok{(diabetes)}
\end{Highlighting}
\end{Shaded}

\begin{verbatim}
## Rows: 768
## Columns: 9
## $ Pregnancies              <dbl> 6, 1, 8, 1, 0, 5, 3, 10, 2, 8, 4, 10, 10, 1, ~
## $ Glucose                  <dbl> 148, 85, 183, 89, 137, 116, 78, 115, 197, 125~
## $ BloodPressure            <dbl> 72, 66, 64, 66, 40, 74, 50, 0, 70, 96, 92, 74~
## $ SkinThickness            <dbl> 35, 29, 0, 23, 35, 0, 32, 0, 45, 0, 0, 0, 0, ~
## $ Insulin                  <dbl> 0, 0, 0, 94, 168, 0, 88, 0, 543, 0, 0, 0, 0, ~
## $ BMI                      <dbl> 33.6, 26.6, 23.3, 28.1, 43.1, 25.6, 31.0, 35.~
## $ DiabetesPedigreeFunction <dbl> 0.627, 0.351, 0.672, 0.167, 2.288, 0.201, 0.2~
## $ Age                      <dbl> 50, 31, 32, 21, 33, 30, 26, 29, 53, 54, 30, 3~
## $ Outcome                  <dbl> 1, 0, 1, 0, 1, 0, 1, 0, 1, 1, 0, 1, 0, 1, 1, ~
\end{verbatim}

\begin{Shaded}
\begin{Highlighting}[]
\NormalTok{skimr}\SpecialCharTok{::}\FunctionTok{skim}\NormalTok{(diabetes)}
\end{Highlighting}
\end{Shaded}

\begin{longtable}[]{@{}ll@{}}
\caption{Data summary}\tabularnewline
\toprule\noalign{}
\endfirsthead
\endhead
\bottomrule\noalign{}
\endlastfoot
Name & diabetes \\
Number of rows & 768 \\
Number of columns & 9 \\
\_\_\_\_\_\_\_\_\_\_\_\_\_\_\_\_\_\_\_\_\_\_\_ & \\
Column type frequency: & \\
numeric & 9 \\
\_\_\_\_\_\_\_\_\_\_\_\_\_\_\_\_\_\_\_\_\_\_\_\_ & \\
Group variables & None \\
\end{longtable}

\textbf{Variable type: numeric}

\begin{longtable}[]{@{}
  >{\raggedright\arraybackslash}p{(\columnwidth - 20\tabcolsep) * \real{0.2451}}
  >{\raggedleft\arraybackslash}p{(\columnwidth - 20\tabcolsep) * \real{0.0980}}
  >{\raggedleft\arraybackslash}p{(\columnwidth - 20\tabcolsep) * \real{0.1373}}
  >{\raggedleft\arraybackslash}p{(\columnwidth - 20\tabcolsep) * \real{0.0686}}
  >{\raggedleft\arraybackslash}p{(\columnwidth - 20\tabcolsep) * \real{0.0686}}
  >{\raggedleft\arraybackslash}p{(\columnwidth - 20\tabcolsep) * \real{0.0588}}
  >{\raggedleft\arraybackslash}p{(\columnwidth - 20\tabcolsep) * \real{0.0588}}
  >{\raggedleft\arraybackslash}p{(\columnwidth - 20\tabcolsep) * \real{0.0686}}
  >{\raggedleft\arraybackslash}p{(\columnwidth - 20\tabcolsep) * \real{0.0686}}
  >{\raggedleft\arraybackslash}p{(\columnwidth - 20\tabcolsep) * \real{0.0686}}
  >{\raggedright\arraybackslash}p{(\columnwidth - 20\tabcolsep) * \real{0.0588}}@{}}
\toprule\noalign{}
\begin{minipage}[b]{\linewidth}\raggedright
skim\_variable
\end{minipage} & \begin{minipage}[b]{\linewidth}\raggedleft
n\_missing
\end{minipage} & \begin{minipage}[b]{\linewidth}\raggedleft
complete\_rate
\end{minipage} & \begin{minipage}[b]{\linewidth}\raggedleft
mean
\end{minipage} & \begin{minipage}[b]{\linewidth}\raggedleft
sd
\end{minipage} & \begin{minipage}[b]{\linewidth}\raggedleft
p0
\end{minipage} & \begin{minipage}[b]{\linewidth}\raggedleft
p25
\end{minipage} & \begin{minipage}[b]{\linewidth}\raggedleft
p50
\end{minipage} & \begin{minipage}[b]{\linewidth}\raggedleft
p75
\end{minipage} & \begin{minipage}[b]{\linewidth}\raggedleft
p100
\end{minipage} & \begin{minipage}[b]{\linewidth}\raggedright
hist
\end{minipage} \\
\midrule\noalign{}
\endhead
\bottomrule\noalign{}
\endlastfoot
Pregnancies & 0 & 1 & 3.85 & 3.37 & 0.00 & 1.00 & 3.00 & 6.00 & 17.00 &
▇▃▂▁▁ \\
Glucose & 0 & 1 & 120.89 & 31.97 & 0.00 & 99.00 & 117.00 & 140.25 &
199.00 & ▁▁▇▆▂ \\
BloodPressure & 0 & 1 & 69.11 & 19.36 & 0.00 & 62.00 & 72.00 & 80.00 &
122.00 & ▁▁▇▇▁ \\
SkinThickness & 0 & 1 & 20.54 & 15.95 & 0.00 & 0.00 & 23.00 & 32.00 &
99.00 & ▇▇▂▁▁ \\
Insulin & 0 & 1 & 79.80 & 115.24 & 0.00 & 0.00 & 30.50 & 127.25 & 846.00
& ▇▁▁▁▁ \\
BMI & 0 & 1 & 31.99 & 7.88 & 0.00 & 27.30 & 32.00 & 36.60 & 67.10 &
▁▃▇▂▁ \\
DiabetesPedigreeFunction & 0 & 1 & 0.47 & 0.33 & 0.08 & 0.24 & 0.37 &
0.63 & 2.42 & ▇▃▁▁▁ \\
Age & 0 & 1 & 33.24 & 11.76 & 21.00 & 24.00 & 29.00 & 41.00 & 81.00 &
▇▃▁▁▁ \\
Outcome & 0 & 1 & 0.35 & 0.48 & 0.00 & 0.00 & 0.00 & 1.00 & 1.00 &
▇▁▁▁▅ \\
\end{longtable}

\begin{Shaded}
\begin{Highlighting}[]
\CommentTok{\# Outcome balance}
\FunctionTok{table}\NormalTok{(diabetes}\SpecialCharTok{$}\NormalTok{Outcome) }\SpecialCharTok{\%\textgreater{}\%}
  \FunctionTok{as.data.frame}\NormalTok{() }\SpecialCharTok{\%\textgreater{}\%}
  \FunctionTok{set\_names}\NormalTok{(}\FunctionTok{c}\NormalTok{(}\StringTok{"Outcome"}\NormalTok{,}\StringTok{"Count"}\NormalTok{)) }\SpecialCharTok{\%\textgreater{}\%}
  \FunctionTok{kable}\NormalTok{() }\SpecialCharTok{\%\textgreater{}\%}
  \FunctionTok{kable\_styling}\NormalTok{(}\AttributeTok{full\_width =} \ConstantTok{FALSE}\NormalTok{)}
\end{Highlighting}
\end{Shaded}

\begin{longtable}[t]{lr}
\toprule
Outcome & Count\\
\midrule
0 & 500\\
1 & 268\\
\bottomrule
\end{longtable}

Estimated output description: - glimpse: shows 9 columns (Pregnancies,
Glucose, BloodPressure, SkinThickness, Insulin, BMI,
DiabetesPedigreeFunction, Age, Outcome). - skim: shows missing counts
and summary stats. - Outcome table: counts of class 0 vs 1 (approx 500
vs 268 per the dataset description).

Optional: fetch dataset metadata from OpenML (eval=FALSE)

\begin{Shaded}
\begin{Highlighting}[]
\CommentTok{\# If internet is available, use the OpenML package to fetch metadata and description}
\CommentTok{\# library(OpenML)}
\CommentTok{\# openml\_ds \textless{}{-} getOMLDataSet(data.id = 37)}
\CommentTok{\# cat(openml\_ds$description)}
\CommentTok{\# saveRDS(openml\_ds, file = "data/openml\_ds\_37.rds")}
\end{Highlighting}
\end{Shaded}

\subsection{Data cleaning \& transformations
(scrub)}\label{data-cleaning-transformations-scrub}

Description: This section implements the key transformations required by
the rubric: - Replace biologically implausible zeros in some clinical
fields with NA. - Show missingness pattern. - Create derived features:
age groups, log-transformed Insulin, BMI categories. - Demonstrate wide
-\textgreater{} long transformation for plotting.

We also create two imputation strategies: 1) Simple median imputation
(fast, reproducible). 2) Multiple imputation (mice) --- shown but
commented as optional heavier alternative.

Expected output: - Missing value counts before and after transformation.
- A small preview of the cleaned data.

\begin{Shaded}
\begin{Highlighting}[]
\CommentTok{\# Columns where 0 is not biologically plausible and should be NA}
\NormalTok{zero\_as\_na\_cols }\OtherTok{\textless{}{-}} \FunctionTok{c}\NormalTok{(}\StringTok{"Glucose"}\NormalTok{,}\StringTok{"BloodPressure"}\NormalTok{,}\StringTok{"SkinThickness"}\NormalTok{,}\StringTok{"Insulin"}\NormalTok{,}\StringTok{"BMI"}\NormalTok{)}

\CommentTok{\# Replace zeros with NA}
\NormalTok{diabetes\_clean }\OtherTok{\textless{}{-}}\NormalTok{ diabetes }\SpecialCharTok{\%\textgreater{}\%}
  \FunctionTok{mutate}\NormalTok{(}\FunctionTok{across}\NormalTok{(}\FunctionTok{all\_of}\NormalTok{(zero\_as\_na\_cols), }\SpecialCharTok{\textasciitilde{}} \FunctionTok{na\_if}\NormalTok{(., }\DecValTok{0}\NormalTok{)))}

\CommentTok{\# Show missing counts}
\NormalTok{missing\_counts }\OtherTok{\textless{}{-}} \FunctionTok{sapply}\NormalTok{(diabetes\_clean, }\ControlFlowTok{function}\NormalTok{(x) }\FunctionTok{sum}\NormalTok{(}\FunctionTok{is.na}\NormalTok{(x)))}
\NormalTok{missing\_counts\_df }\OtherTok{\textless{}{-}} \FunctionTok{tibble}\NormalTok{(}\AttributeTok{variable =} \FunctionTok{names}\NormalTok{(missing\_counts), }\AttributeTok{missing =} \FunctionTok{as.integer}\NormalTok{(missing\_counts))}
\NormalTok{missing\_counts\_df }\SpecialCharTok{\%\textgreater{}\%}
  \FunctionTok{kable}\NormalTok{() }\SpecialCharTok{\%\textgreater{}\%}
  \FunctionTok{kable\_styling}\NormalTok{(}\AttributeTok{full\_width =} \ConstantTok{FALSE}\NormalTok{)}
\end{Highlighting}
\end{Shaded}

\begin{longtable}[t]{lr}
\toprule
variable & missing\\
\midrule
Pregnancies & 0\\
Glucose & 5\\
BloodPressure & 35\\
SkinThickness & 227\\
Insulin & 374\\
\addlinespace
BMI & 11\\
DiabetesPedigreeFunction & 0\\
Age & 0\\
Outcome & 0\\
\bottomrule
\end{longtable}

\begin{Shaded}
\begin{Highlighting}[]
\CommentTok{\# Create derived features}
\NormalTok{diabetes\_clean }\OtherTok{\textless{}{-}}\NormalTok{ diabetes\_clean }\SpecialCharTok{\%\textgreater{}\%}
  \FunctionTok{mutate}\NormalTok{(}
    \AttributeTok{AgeGroup =} \FunctionTok{case\_when}\NormalTok{(}
\NormalTok{      Age }\SpecialCharTok{\textless{}=} \DecValTok{30} \SpecialCharTok{\textasciitilde{}} \StringTok{"21{-}30"}\NormalTok{,}
\NormalTok{      Age }\SpecialCharTok{\textless{}=} \DecValTok{40} \SpecialCharTok{\textasciitilde{}} \StringTok{"31{-}40"}\NormalTok{,}
\NormalTok{      Age }\SpecialCharTok{\textless{}=} \DecValTok{50} \SpecialCharTok{\textasciitilde{}} \StringTok{"41{-}50"}\NormalTok{,}
      \ConstantTok{TRUE}      \SpecialCharTok{\textasciitilde{}} \StringTok{"51+"}
\NormalTok{    ),}
    \AttributeTok{Insulin\_log =} \FunctionTok{log1p}\NormalTok{(Insulin),   }\CommentTok{\# log(Insulin + 1)}
    \AttributeTok{BMI\_cat =} \FunctionTok{cut}\NormalTok{(BMI, }\AttributeTok{breaks =} \FunctionTok{c}\NormalTok{(}\DecValTok{0}\NormalTok{,}\FloatTok{18.5}\NormalTok{,}\DecValTok{25}\NormalTok{,}\DecValTok{30}\NormalTok{,}\ConstantTok{Inf}\NormalTok{), }\AttributeTok{labels =} \FunctionTok{c}\NormalTok{(}\StringTok{"Underweight"}\NormalTok{,}\StringTok{"Normal"}\NormalTok{,}\StringTok{"Overweight"}\NormalTok{,}\StringTok{"Obese"}\NormalTok{))}
\NormalTok{  )}

\CommentTok{\# Show top rows}
\FunctionTok{head}\NormalTok{(diabetes\_clean) }\SpecialCharTok{\%\textgreater{}\%} \FunctionTok{kable}\NormalTok{() }\SpecialCharTok{\%\textgreater{}\%} \FunctionTok{kable\_styling}\NormalTok{(}\AttributeTok{full\_width =} \ConstantTok{FALSE}\NormalTok{)}
\end{Highlighting}
\end{Shaded}

\begin{longtable}[t]{rrrrrrrrrlrl}
\toprule
Pregnancies & Glucose & BloodPressure & SkinThickness & Insulin & BMI & DiabetesPedigreeFunction & Age & Outcome & AgeGroup & Insulin\_log & BMI\_cat\\
\midrule
6 & 148 & 72 & 35 & NA & 33.6 & 0.627 & 50 & 1 & 41-50 & NA & Obese\\
1 & 85 & 66 & 29 & NA & 26.6 & 0.351 & 31 & 0 & 31-40 & NA & Overweight\\
8 & 183 & 64 & NA & NA & 23.3 & 0.672 & 32 & 1 & 31-40 & NA & Normal\\
1 & 89 & 66 & 23 & 94 & 28.1 & 0.167 & 21 & 0 & 21-30 & 4.553877 & Overweight\\
0 & 137 & 40 & 35 & 168 & 43.1 & 2.288 & 33 & 1 & 31-40 & 5.129899 & Obese\\
\addlinespace
5 & 116 & 74 & NA & NA & 25.6 & 0.201 & 30 & 0 & 21-30 & NA & Overweight\\
\bottomrule
\end{longtable}

Estimated output description: - missing\_counts will show considerable
missingness in Insulin, SkinThickness, sometimes Glucose or BMI
depending on zeros. - Derived features appear in the data frame:
AgeGroup, Insulin\_log, BMI\_cat.

\subsection{Simple imputation (median) --- reproducible
baseline}\label{simple-imputation-median-reproducible-baseline}

Description: We perform a simple median imputation for numeric
predictors to make initial models reproducible. We keep a missingness
indicator column for each imputed variable to preserve missingness
information (useful for modeling).

Expected output: - A summary table showing replaced values count. - A
glimpse of imputed dataset.

\begin{Shaded}
\begin{Highlighting}[]
\NormalTok{numeric\_vars }\OtherTok{\textless{}{-}} \FunctionTok{c}\NormalTok{(}\StringTok{"Pregnancies"}\NormalTok{,}\StringTok{"Glucose"}\NormalTok{,}\StringTok{"BloodPressure"}\NormalTok{,}\StringTok{"SkinThickness"}\NormalTok{,}\StringTok{"Insulin"}\NormalTok{,}\StringTok{"BMI"}\NormalTok{,}\StringTok{"DiabetesPedigreeFunction"}\NormalTok{,}\StringTok{"Age"}\NormalTok{)}

\NormalTok{diabetes\_imputed }\OtherTok{\textless{}{-}}\NormalTok{ diabetes\_clean}

\CommentTok{\# Create missingness indicators and median{-}impute}
\ControlFlowTok{for}\NormalTok{ (v }\ControlFlowTok{in}\NormalTok{ zero\_as\_na\_cols) \{}
\NormalTok{  mi\_col }\OtherTok{\textless{}{-}} \FunctionTok{paste0}\NormalTok{(v, }\StringTok{"\_na"}\NormalTok{)}
\NormalTok{  diabetes\_imputed[[mi\_col]] }\OtherTok{\textless{}{-}} \FunctionTok{is.na}\NormalTok{(diabetes\_imputed[[v]])}
\NormalTok{  med }\OtherTok{\textless{}{-}} \FunctionTok{median}\NormalTok{(diabetes\_imputed[[v]], }\AttributeTok{na.rm =} \ConstantTok{TRUE}\NormalTok{)}
\NormalTok{  diabetes\_imputed[[v]][}\FunctionTok{is.na}\NormalTok{(diabetes\_imputed[[v]])] }\OtherTok{\textless{}{-}}\NormalTok{ med}
\NormalTok{\}}

\CommentTok{\# Quick check}
\FunctionTok{sapply}\NormalTok{(diabetes\_imputed, }\ControlFlowTok{function}\NormalTok{(x) }\FunctionTok{sum}\NormalTok{(}\FunctionTok{is.na}\NormalTok{(x))) }\SpecialCharTok{\%\textgreater{}\%} \FunctionTok{head}\NormalTok{(}\DecValTok{20}\NormalTok{)}
\end{Highlighting}
\end{Shaded}

\begin{verbatim}
##              Pregnancies                  Glucose            BloodPressure 
##                        0                        0                        0 
##            SkinThickness                  Insulin                      BMI 
##                        0                        0                        0 
## DiabetesPedigreeFunction                      Age                  Outcome 
##                        0                        0                        0 
##                 AgeGroup              Insulin_log                  BMI_cat 
##                        0                      374                       11 
##               Glucose_na         BloodPressure_na         SkinThickness_na 
##                        0                        0                        0 
##               Insulin_na                   BMI_na 
##                        0                        0
\end{verbatim}

Estimated output description: - After this chunk, numeric predictor NAs
are replaced; missingness indicator columns show TRUE for originally
missing rows. No NA remains in numeric\_vars.

Optional: multiple imputation with mice (eval=FALSE)

\begin{Shaded}
\begin{Highlighting}[]
\CommentTok{\# Multiple imputation is more statistically principled. This chunk is optional (longer runtime).}
\CommentTok{\# mi \textless{}{-} mice(diabetes\_clean \%\textgreater{}\% select(all\_of(numeric\_vars)), m = 5, seed = 123)}
\CommentTok{\# completed \textless{}{-} complete(mi, action = "long", include = FALSE)}
\CommentTok{\# Use completed data for sensitivity analyses and to compare final model stability.}
\end{Highlighting}
\end{Shaded}

\subsection{Exploratory data analysis
(explore)}\label{exploratory-data-analysis-explore}

Description: We generate descriptive plots required by rubric: -
Distribution plots (density/histogram) by Outcome. - Correlation matrix
of numeric predictors. - Boxplots for strong predictors (Glucose, BMI).
- Table summarizing outcome by age group.

We show what the expected visual patterns might be (e.g., higher Glucose
in Outcome=1).

Expected output: - Density plots for Glucose, BMI, Insulin etc. with
clear separation for Outcome. - Correlation heatmap (shows Glucose
correlated with Outcome, maybe BMI somewhat correlated).

\begin{Shaded}
\begin{Highlighting}[]
\FunctionTok{library}\NormalTok{(tidyverse)}

\CommentTok{\# Ensure the imputed dataset exists}
\ControlFlowTok{if}\NormalTok{ (}\SpecialCharTok{!}\FunctionTok{exists}\NormalTok{(}\StringTok{"diabetes\_imputed"}\NormalTok{)) }\FunctionTok{stop}\NormalTok{(}\StringTok{"diabetes\_imputed not found. Run the cleaning/imputation chunk first."}\NormalTok{)}

\NormalTok{plot\_vars }\OtherTok{\textless{}{-}} \FunctionTok{c}\NormalTok{(}\StringTok{"Glucose"}\NormalTok{,}\StringTok{"BMI"}\NormalTok{,}\StringTok{"Insulin"}\NormalTok{,}\StringTok{"BloodPressure"}\NormalTok{,}\StringTok{"SkinThickness"}\NormalTok{,}\StringTok{"DiabetesPedigreeFunction"}\NormalTok{)}
\NormalTok{di\_long }\OtherTok{\textless{}{-}}\NormalTok{ diabetes\_imputed }\SpecialCharTok{\%\textgreater{}\%} \FunctionTok{pivot\_longer}\NormalTok{(}\AttributeTok{cols =} \FunctionTok{all\_of}\NormalTok{(plot\_vars), }\AttributeTok{names\_to =} \StringTok{"measure"}\NormalTok{, }\AttributeTok{values\_to =} \StringTok{"value"}\NormalTok{)}

\CommentTok{\# Create ggplots}
\NormalTok{p1 }\OtherTok{\textless{}{-}} \FunctionTok{ggplot}\NormalTok{(di\_long }\SpecialCharTok{\%\textgreater{}\%} \FunctionTok{filter}\NormalTok{(measure }\SpecialCharTok{==} \StringTok{"Glucose"}\NormalTok{),}
             \FunctionTok{aes}\NormalTok{(}\AttributeTok{x =}\NormalTok{ value, }\AttributeTok{fill =} \FunctionTok{factor}\NormalTok{(Outcome))) }\SpecialCharTok{+}
  \FunctionTok{geom\_density}\NormalTok{(}\AttributeTok{alpha =} \FloatTok{0.4}\NormalTok{) }\SpecialCharTok{+}
  \FunctionTok{labs}\NormalTok{(}\AttributeTok{title =} \StringTok{"Glucose distribution by Outcome"}\NormalTok{, }\AttributeTok{x =} \StringTok{"Glucose"}\NormalTok{, }\AttributeTok{fill =} \StringTok{"Outcome"}\NormalTok{) }\SpecialCharTok{+}
  \FunctionTok{theme\_minimal}\NormalTok{()}

\NormalTok{p2 }\OtherTok{\textless{}{-}} \FunctionTok{ggplot}\NormalTok{(di\_long }\SpecialCharTok{\%\textgreater{}\%} \FunctionTok{filter}\NormalTok{(measure }\SpecialCharTok{==} \StringTok{"BMI"}\NormalTok{),}
             \FunctionTok{aes}\NormalTok{(}\AttributeTok{x =}\NormalTok{ value, }\AttributeTok{fill =} \FunctionTok{factor}\NormalTok{(Outcome))) }\SpecialCharTok{+}
  \FunctionTok{geom\_density}\NormalTok{(}\AttributeTok{alpha =} \FloatTok{0.4}\NormalTok{) }\SpecialCharTok{+}
  \FunctionTok{labs}\NormalTok{(}\AttributeTok{title =} \StringTok{"BMI distribution by Outcome"}\NormalTok{, }\AttributeTok{x =} \StringTok{"BMI"}\NormalTok{, }\AttributeTok{fill =} \StringTok{"Outcome"}\NormalTok{) }\SpecialCharTok{+}
  \FunctionTok{theme\_minimal}\NormalTok{()}

\NormalTok{p3 }\OtherTok{\textless{}{-}} \FunctionTok{ggplot}\NormalTok{(di\_long }\SpecialCharTok{\%\textgreater{}\%} \FunctionTok{filter}\NormalTok{(measure }\SpecialCharTok{==} \StringTok{"Insulin"}\NormalTok{),}
             \FunctionTok{aes}\NormalTok{(}\AttributeTok{x =}\NormalTok{ value, }\AttributeTok{fill =} \FunctionTok{factor}\NormalTok{(Outcome))) }\SpecialCharTok{+}
  \FunctionTok{geom\_density}\NormalTok{(}\AttributeTok{alpha =} \FloatTok{0.4}\NormalTok{) }\SpecialCharTok{+}
  \FunctionTok{labs}\NormalTok{(}\AttributeTok{title =} \StringTok{"Insulin distribution by Outcome"}\NormalTok{, }\AttributeTok{x =} \StringTok{"Insulin"}\NormalTok{, }\AttributeTok{fill =} \StringTok{"Outcome"}\NormalTok{) }\SpecialCharTok{+}
  \FunctionTok{theme\_minimal}\NormalTok{()}

\CommentTok{\# Sanity check: ensure these are ggplot objects (accept either "gg" or "ggplot" class)}
\NormalTok{is\_ggplot\_obj }\OtherTok{\textless{}{-}} \ControlFlowTok{function}\NormalTok{(obj) \{}
  \FunctionTok{inherits}\NormalTok{(obj, }\StringTok{"gg"}\NormalTok{) }\SpecialCharTok{||} \FunctionTok{inherits}\NormalTok{(obj, }\StringTok{"ggplot"}\NormalTok{)}
\NormalTok{\}}
\ControlFlowTok{if}\NormalTok{ (}\SpecialCharTok{!}\FunctionTok{all}\NormalTok{(}\FunctionTok{sapply}\NormalTok{(}\FunctionTok{list}\NormalTok{(p1, p2, p3), is\_ggplot\_obj))) \{}
  \FunctionTok{stop}\NormalTok{(}\StringTok{"One of the plot objects is not a ggplot. Recreate p1/p2/p3."}\NormalTok{)}
\NormalTok{\}}

\CommentTok{\# Try patchwork first (preferred); if it fails, fallback to cowplot or gridExtra}
\NormalTok{combined\_plot }\OtherTok{\textless{}{-}} \ConstantTok{NULL}
\NormalTok{can\_use\_patchwork }\OtherTok{\textless{}{-}} \ConstantTok{FALSE}
\ControlFlowTok{if}\NormalTok{ (}\StringTok{"ggplot2"} \SpecialCharTok{\%in\%} \FunctionTok{.packages}\NormalTok{() }\SpecialCharTok{||} \StringTok{"package:ggplot2"} \SpecialCharTok{\%in\%} \FunctionTok{search}\NormalTok{()) \{}
\NormalTok{  can\_use\_patchwork }\OtherTok{\textless{}{-}} \FunctionTok{exists}\NormalTok{(}\StringTok{"is\_ggplot"}\NormalTok{, }\AttributeTok{where =} \FunctionTok{asNamespace}\NormalTok{(}\StringTok{"ggplot2"}\NormalTok{), }\AttributeTok{inherits =} \ConstantTok{FALSE}\NormalTok{)}
\NormalTok{\}}
\ControlFlowTok{if}\NormalTok{ (can\_use\_patchwork) \{}
  \FunctionTok{try}\NormalTok{(\{}
    \FunctionTok{library}\NormalTok{(patchwork)}
\NormalTok{    combined\_plot }\OtherTok{\textless{}{-}}\NormalTok{ (p1 }\SpecialCharTok{|}\NormalTok{ p2) }\SpecialCharTok{/}\NormalTok{ p3}
    \FunctionTok{print}\NormalTok{(combined\_plot)}
\NormalTok{  \}, }\AttributeTok{silent =} \ConstantTok{TRUE}\NormalTok{)}
\NormalTok{\}}

\ControlFlowTok{if}\NormalTok{ (}\FunctionTok{is.null}\NormalTok{(combined\_plot)) \{}
  \FunctionTok{message}\NormalTok{(}\StringTok{"patchwork unavailable or incompatible. Using fallback layout (cowplot or gridExtra)."}\NormalTok{)}
  \ControlFlowTok{if}\NormalTok{ (}\FunctionTok{requireNamespace}\NormalTok{(}\StringTok{"cowplot"}\NormalTok{, }\AttributeTok{quietly =} \ConstantTok{TRUE}\NormalTok{)) \{}
    \FunctionTok{library}\NormalTok{(cowplot)}
\NormalTok{    top\_row }\OtherTok{\textless{}{-}}\NormalTok{ cowplot}\SpecialCharTok{::}\FunctionTok{plot\_grid}\NormalTok{(p1, p2, }\AttributeTok{ncol =} \DecValTok{2}\NormalTok{, }\AttributeTok{align =} \StringTok{"hv"}\NormalTok{)}
\NormalTok{    final\_plot }\OtherTok{\textless{}{-}}\NormalTok{ cowplot}\SpecialCharTok{::}\FunctionTok{plot\_grid}\NormalTok{(top\_row, p3, }\AttributeTok{ncol =} \DecValTok{1}\NormalTok{, }\AttributeTok{rel\_heights =} \FunctionTok{c}\NormalTok{(}\DecValTok{1}\NormalTok{, }\FloatTok{0.8}\NormalTok{))}
    \FunctionTok{print}\NormalTok{(final\_plot)}
\NormalTok{  \} }\ControlFlowTok{else}\NormalTok{ \{}
    \ControlFlowTok{if}\NormalTok{ (}\SpecialCharTok{!}\FunctionTok{requireNamespace}\NormalTok{(}\StringTok{"gridExtra"}\NormalTok{, }\AttributeTok{quietly =} \ConstantTok{TRUE}\NormalTok{)) }\FunctionTok{install.packages}\NormalTok{(}\StringTok{"gridExtra"}\NormalTok{)}
    \FunctionTok{library}\NormalTok{(gridExtra)}
    \FunctionTok{library}\NormalTok{(grid)}
\NormalTok{    layout\_mat }\OtherTok{\textless{}{-}} \FunctionTok{rbind}\NormalTok{(}\FunctionTok{c}\NormalTok{(}\DecValTok{1}\NormalTok{,}\DecValTok{2}\NormalTok{), }\FunctionTok{c}\NormalTok{(}\DecValTok{3}\NormalTok{,}\DecValTok{3}\NormalTok{))}
    \FunctionTok{grid.arrange}\NormalTok{(}\AttributeTok{grobs =} \FunctionTok{list}\NormalTok{(p1, p2, p3),}
                 \AttributeTok{layout\_matrix =}\NormalTok{ layout\_mat,}
                 \AttributeTok{top =} \FunctionTok{textGrob}\NormalTok{(}\StringTok{"Predictor distributions by Outcome"}\NormalTok{, }\AttributeTok{gp =} \FunctionTok{gpar}\NormalTok{(}\AttributeTok{fontsize =} \DecValTok{14}\NormalTok{, }\AttributeTok{fontface =} \StringTok{"bold"}\NormalTok{)))}
\NormalTok{  \}}
\NormalTok{\}}
\end{Highlighting}
\end{Shaded}

\begin{verbatim}
## patchwork unavailable or incompatible. Using fallback layout (cowplot or gridExtra).
\end{verbatim}

\begin{verbatim}
## Warning: package 'cowplot' was built under R version 4.4.2
\end{verbatim}

\begin{verbatim}
## 
## Attaching package: 'cowplot'
\end{verbatim}

\begin{verbatim}
## The following object is masked from 'package:lubridate':
## 
##     stamp
\end{verbatim}

\includegraphics{Final-Projet_files/figure-latex/eda-plots-1.pdf}

Estimated output description: - Glucose density: Outcome=1 (diabetes)
shifted to the right (higher glucose). - BMI density: modest shift;
higher BMI for some Outcome=1 cases. - Insulin density: skewed
distribution; transformation helped earlier.

Correlation matrix

\begin{Shaded}
\begin{Highlighting}[]
\NormalTok{num\_df }\OtherTok{\textless{}{-}}\NormalTok{ diabetes\_imputed }\SpecialCharTok{\%\textgreater{}\%} \FunctionTok{select}\NormalTok{(}\FunctionTok{all\_of}\NormalTok{(plot\_vars))}
\NormalTok{corr\_mat }\OtherTok{\textless{}{-}} \FunctionTok{cor}\NormalTok{(num\_df, }\AttributeTok{use =} \StringTok{"complete.obs"}\NormalTok{)}
\NormalTok{corrplot}\SpecialCharTok{::}\FunctionTok{corrplot}\NormalTok{(corr\_mat, }\AttributeTok{method =} \StringTok{"color"}\NormalTok{, }\AttributeTok{tl.cex =} \FloatTok{0.8}\NormalTok{)}
\end{Highlighting}
\end{Shaded}

\includegraphics{Final-Projet_files/figure-latex/correlation-1.pdf}

Estimated output description: - Correlation heatmap shows relationships
among Glucose, BMI, Insulin, and others. Expect moderate correlation
between BMI and SkinThickness; Glucose may not be strongly correlated
with BMI but predictive for Outcome.

\subsection{Data splitting (train/test)}\label{data-splitting-traintest}

Description: Create reproducible training and test sets (75\% train /
25\% test stratified by Outcome) for final evaluation to mimic realistic
hold-out practice. Use caret's createDataPartition for stratification.

Expected output: - Table showing train/test counts and Outcome
distribution preserved.

\begin{Shaded}
\begin{Highlighting}[]
\FunctionTok{set.seed}\NormalTok{(}\DecValTok{123}\NormalTok{)}
\NormalTok{train\_index }\OtherTok{\textless{}{-}} \FunctionTok{createDataPartition}\NormalTok{(diabetes\_imputed}\SpecialCharTok{$}\NormalTok{Outcome, }\AttributeTok{p =} \FloatTok{0.75}\NormalTok{, }\AttributeTok{list =} \ConstantTok{FALSE}\NormalTok{)}
\NormalTok{train }\OtherTok{\textless{}{-}}\NormalTok{ diabetes\_imputed[train\_index, ]}
\NormalTok{test  }\OtherTok{\textless{}{-}}\NormalTok{ diabetes\_imputed[}\SpecialCharTok{{-}}\NormalTok{train\_index, ]}

\CommentTok{\# Verify distribution}
\FunctionTok{bind\_rows}\NormalTok{(}
\NormalTok{  train }\SpecialCharTok{\%\textgreater{}\%} \FunctionTok{summarise}\NormalTok{(}\AttributeTok{n =} \FunctionTok{n}\NormalTok{(), }\AttributeTok{outcome1 =} \FunctionTok{sum}\NormalTok{(Outcome}\SpecialCharTok{==}\DecValTok{1}\NormalTok{)),}
\NormalTok{  test  }\SpecialCharTok{\%\textgreater{}\%} \FunctionTok{summarise}\NormalTok{(}\AttributeTok{n =} \FunctionTok{n}\NormalTok{(), }\AttributeTok{outcome1 =} \FunctionTok{sum}\NormalTok{(Outcome}\SpecialCharTok{==}\DecValTok{1}\NormalTok{))}
\NormalTok{) }\SpecialCharTok{\%\textgreater{}\%} \FunctionTok{mutate}\NormalTok{(}\AttributeTok{outcome1\_rate =}\NormalTok{ outcome1 }\SpecialCharTok{/}\NormalTok{ n) }\SpecialCharTok{\%\textgreater{}\%} \FunctionTok{kable}\NormalTok{()}
\end{Highlighting}
\end{Shaded}

\begin{longtable}[]{@{}rrr@{}}
\toprule\noalign{}
n & outcome1 & outcome1\_rate \\
\midrule\noalign{}
\endhead
\bottomrule\noalign{}
\endlastfoot
576 & 201 & 0.3489583 \\
192 & 67 & 0.3489583 \\
\end{longtable}

\subsection{Baseline model --- logistic
regression}\label{baseline-model-logistic-regression}

Description: Fit a baseline logistic regression with all predictors
(after imputation) to provide an interpretable benchmark. We'll present
coefficients, p-values, and a simple ROC.

Expected output: - Coefficient table with odds ratios. - AUC for the
logistic model on the test set (expected moderate AUC
\textasciitilde0.75 for this dataset historically).

\begin{Shaded}
\begin{Highlighting}[]
\NormalTok{glm\_fit }\OtherTok{\textless{}{-}} \FunctionTok{glm}\NormalTok{(Outcome }\SpecialCharTok{\textasciitilde{}}\NormalTok{ Pregnancies }\SpecialCharTok{+}\NormalTok{ Glucose }\SpecialCharTok{+}\NormalTok{ BloodPressure }\SpecialCharTok{+}\NormalTok{ SkinThickness }\SpecialCharTok{+}\NormalTok{ Insulin }\SpecialCharTok{+}\NormalTok{ BMI }\SpecialCharTok{+}
\NormalTok{                 DiabetesPedigreeFunction }\SpecialCharTok{+}\NormalTok{ Age, }\AttributeTok{data =}\NormalTok{ train, }\AttributeTok{family =}\NormalTok{ binomial)}

\FunctionTok{summary}\NormalTok{(glm\_fit)}
\end{Highlighting}
\end{Shaded}

\begin{verbatim}
## 
## Call:
## glm(formula = Outcome ~ Pregnancies + Glucose + BloodPressure + 
##     SkinThickness + Insulin + BMI + DiabetesPedigreeFunction + 
##     Age, family = binomial, data = train)
## 
## Coefficients:
##                           Estimate Std. Error z value             Pr(>|z|)    
## (Intercept)              -9.419723   0.951095  -9.904 < 0.0000000000000002 ***
## Pregnancies               0.116806   0.037999   3.074              0.00211 ** 
## Glucose                   0.040483   0.004728   8.562 < 0.0000000000000002 ***
## BloodPressure            -0.008729   0.010045  -0.869              0.38482    
## SkinThickness             0.009409   0.015390   0.611              0.54097    
## Insulin                  -0.001247   0.001273  -0.979              0.32735    
## BMI                       0.090991   0.020999   4.333            0.0000147 ***
## DiabetesPedigreeFunction  0.661885   0.331211   1.998              0.04568 *  
## Age                       0.012160   0.011164   1.089              0.27606    
## ---
## Signif. codes:  0 '***' 0.001 '**' 0.01 '*' 0.05 '.' 0.1 ' ' 1
## 
## (Dispersion parameter for binomial family taken to be 1)
## 
##     Null deviance: 745.11  on 575  degrees of freedom
## Residual deviance: 524.39  on 567  degrees of freedom
## AIC: 542.39
## 
## Number of Fisher Scoring iterations: 5
\end{verbatim}

\begin{Shaded}
\begin{Highlighting}[]
\CommentTok{\# Odds ratios and 95\% CI}
\NormalTok{exp\_coef }\OtherTok{\textless{}{-}}\NormalTok{ broom}\SpecialCharTok{::}\FunctionTok{tidy}\NormalTok{(glm\_fit) }\SpecialCharTok{\%\textgreater{}\%}
  \FunctionTok{mutate}\NormalTok{(}\AttributeTok{OR =} \FunctionTok{exp}\NormalTok{(estimate),}
         \AttributeTok{OR\_low =} \FunctionTok{exp}\NormalTok{(estimate }\SpecialCharTok{{-}} \FloatTok{1.96} \SpecialCharTok{*}\NormalTok{ std.error),}
         \AttributeTok{OR\_high =} \FunctionTok{exp}\NormalTok{(estimate }\SpecialCharTok{+} \FloatTok{1.96} \SpecialCharTok{*}\NormalTok{ std.error))}
\NormalTok{exp\_coef }\SpecialCharTok{\%\textgreater{}\%} 
  \FunctionTok{select}\NormalTok{(term, estimate, std.error, OR, OR\_low, OR\_high) }\SpecialCharTok{\%\textgreater{}\%} 
  \FunctionTok{kable}\NormalTok{(}
    \AttributeTok{format =} \StringTok{"latex"}\NormalTok{,}
    \AttributeTok{booktabs =} \ConstantTok{TRUE}\NormalTok{,}
    \AttributeTok{escape =} \ConstantTok{TRUE}\NormalTok{,}
    \AttributeTok{longtable =} \ConstantTok{FALSE}
\NormalTok{  ) }\SpecialCharTok{\%\textgreater{}\%}
  \FunctionTok{kable\_styling}\NormalTok{(}\AttributeTok{full\_width =} \ConstantTok{FALSE}\NormalTok{, }\AttributeTok{font\_size =} \DecValTok{9}\NormalTok{)}
\end{Highlighting}
\end{Shaded}

\begin{table}
\centering\begingroup\fontsize{9}{11}\selectfont

\begin{tabular}{lrrrrr}
\toprule
term & estimate & std.error & OR & OR\_low & OR\_high\\
\midrule
(Intercept) & -9.4197232 & 0.9510946 & 0.0000811 & 0.0000126 & 0.0005232\\
Pregnancies & 0.1168057 & 0.0379992 & 1.1239010 & 1.0432358 & 1.2108034\\
Glucose & 0.0404826 & 0.0047280 & 1.0413132 & 1.0317081 & 1.0510077\\
BloodPressure & -0.0087293 & 0.0100446 & 0.9913087 & 0.9719833 & 1.0110184\\
SkinThickness & 0.0094087 & 0.0153902 & 1.0094531 & 0.9794578 & 1.0403671\\
\addlinespace
Insulin & -0.0012471 & 0.0012732 & 0.9987537 & 0.9962643 & 1.0012492\\
BMI & 0.0909910 & 0.0209989 & 1.0952592 & 1.0510957 & 1.1412783\\
DiabetesPedigreeFunction & 0.6618850 & 0.3312111 & 1.9384429 & 1.0127924 & 3.7100997\\
Age & 0.0121602 & 0.0111643 & 1.0122344 & 0.9903253 & 1.0346282\\
\bottomrule
\end{tabular}
\endgroup{}
\end{table}

Evaluation of logistic on test set

\begin{Shaded}
\begin{Highlighting}[]
\CommentTok{\# Predict and compute ROC/AUC}
\NormalTok{test}\SpecialCharTok{$}\NormalTok{pred\_glm }\OtherTok{\textless{}{-}} \FunctionTok{predict}\NormalTok{(glm\_fit, }\AttributeTok{newdata =}\NormalTok{ test, }\AttributeTok{type =} \StringTok{"response"}\NormalTok{)}
\NormalTok{roc\_glm }\OtherTok{\textless{}{-}}\NormalTok{ pROC}\SpecialCharTok{::}\FunctionTok{roc}\NormalTok{(test}\SpecialCharTok{$}\NormalTok{Outcome, test}\SpecialCharTok{$}\NormalTok{pred\_glm)}
\end{Highlighting}
\end{Shaded}

\begin{verbatim}
## Setting levels: control = 0, case = 1
\end{verbatim}

\begin{verbatim}
## Setting direction: controls < cases
\end{verbatim}

\begin{Shaded}
\begin{Highlighting}[]
\NormalTok{pROC}\SpecialCharTok{::}\FunctionTok{auc}\NormalTok{(roc\_glm)}
\end{Highlighting}
\end{Shaded}

\begin{verbatim}
## Area under the curve: 0.8242
\end{verbatim}

\begin{Shaded}
\begin{Highlighting}[]
\CommentTok{\# Plot ROC}
\FunctionTok{plot}\NormalTok{(roc\_glm, }\AttributeTok{main =} \FunctionTok{paste}\NormalTok{(}\StringTok{"Logistic ROC (AUC ="}\NormalTok{, }\FunctionTok{round}\NormalTok{(pROC}\SpecialCharTok{::}\FunctionTok{auc}\NormalTok{(roc\_glm),}\DecValTok{3}\NormalTok{), }\StringTok{")"}\NormalTok{))}
\end{Highlighting}
\end{Shaded}

\includegraphics{Final-Projet_files/figure-latex/logistic-eval-1.pdf}

Expected output description: - Coefficient table: Glucose coefficient
strongly positive; age, BMI might be positive. - Logistic AUC:
historically around 0.75 but varies --- expect AUC between 0.70--0.80. -
ROC plot: demonstrates model discriminative ability.

\subsection{Random Forest model
(caret)}\label{random-forest-model-caret}

Description: Fit a random forest using caret with repeated
cross-validation for tuning. This model often achieves competitive
performance on this dataset.

Expected output: - Cross-validated performance results (AUC) during
tuning. - Final test-set AUC and variable importance plot.

\begin{Shaded}
\begin{Highlighting}[]
\CommentTok{\# caret training with repeated CV}
\NormalTok{ctrl }\OtherTok{\textless{}{-}} \FunctionTok{trainControl}\NormalTok{(}\AttributeTok{method =} \StringTok{"repeatedcv"}\NormalTok{, }\AttributeTok{number =} \DecValTok{5}\NormalTok{, }\AttributeTok{repeats =} \DecValTok{3}\NormalTok{,}
                     \AttributeTok{classProbs =} \ConstantTok{TRUE}\NormalTok{, }\AttributeTok{summaryFunction =}\NormalTok{ twoClassSummary, }\AttributeTok{savePredictions =} \ConstantTok{TRUE}\NormalTok{)}

\CommentTok{\# Convert Outcome to factor with levels "no"/"yes"}
\NormalTok{train\_rf }\OtherTok{\textless{}{-}}\NormalTok{ train }\SpecialCharTok{\%\textgreater{}\%}
  \FunctionTok{mutate}\NormalTok{(}\AttributeTok{Outcome =} \FunctionTok{factor}\NormalTok{(}\FunctionTok{ifelse}\NormalTok{(Outcome}\SpecialCharTok{==}\DecValTok{1}\NormalTok{,}\StringTok{"yes"}\NormalTok{,}\StringTok{"no"}\NormalTok{)))}

\NormalTok{test\_rf }\OtherTok{\textless{}{-}}\NormalTok{ test }\SpecialCharTok{\%\textgreater{}\%}
  \FunctionTok{mutate}\NormalTok{(}\AttributeTok{Outcome =} \FunctionTok{factor}\NormalTok{(}\FunctionTok{ifelse}\NormalTok{(Outcome}\SpecialCharTok{==}\DecValTok{1}\NormalTok{,}\StringTok{"yes"}\NormalTok{,}\StringTok{"no"}\NormalTok{)))}

\FunctionTok{set.seed}\NormalTok{(}\DecValTok{123}\NormalTok{)}
\NormalTok{rf\_fit }\OtherTok{\textless{}{-}} \FunctionTok{train}\NormalTok{(Outcome }\SpecialCharTok{\textasciitilde{}}\NormalTok{ Pregnancies }\SpecialCharTok{+}\NormalTok{ Glucose }\SpecialCharTok{+}\NormalTok{ BloodPressure }\SpecialCharTok{+}\NormalTok{ SkinThickness }\SpecialCharTok{+}\NormalTok{ Insulin }\SpecialCharTok{+}\NormalTok{ BMI }\SpecialCharTok{+}
\NormalTok{                  DiabetesPedigreeFunction }\SpecialCharTok{+}\NormalTok{ Age,}
                \AttributeTok{data =}\NormalTok{ train\_rf,}
                \AttributeTok{method =} \StringTok{"rf"}\NormalTok{,}
                \AttributeTok{metric =} \StringTok{"ROC"}\NormalTok{,}
                \AttributeTok{trControl =}\NormalTok{ ctrl,}
                \AttributeTok{tuneLength =} \DecValTok{5}\NormalTok{)}

\NormalTok{rf\_fit}
\end{Highlighting}
\end{Shaded}

\begin{verbatim}
## Random Forest 
## 
## 576 samples
##   8 predictor
##   2 classes: 'no', 'yes' 
## 
## No pre-processing
## Resampling: Cross-Validated (5 fold, repeated 3 times) 
## Summary of sample sizes: 461, 461, 461, 461, 460, 461, ... 
## Resampling results across tuning parameters:
## 
##   mtry  ROC        Sens       Spec     
##   2     0.8218241  0.8542222  0.5806098
##   3     0.8181949  0.8497778  0.5921545
##   5     0.8119946  0.8408889  0.5821138
##   6     0.8115382  0.8382222  0.5805285
##   8     0.8073100  0.8337778  0.5904472
## 
## ROC was used to select the optimal model using the largest value.
## The final value used for the model was mtry = 2.
\end{verbatim}

Evaluate on test set

\begin{Shaded}
\begin{Highlighting}[]
\CommentTok{\# Predict probabilities and compute ROC}
\NormalTok{test\_rf}\SpecialCharTok{$}\NormalTok{pred\_rf }\OtherTok{\textless{}{-}} \FunctionTok{predict}\NormalTok{(rf\_fit, }\AttributeTok{newdata =}\NormalTok{ test\_rf, }\AttributeTok{type =} \StringTok{"prob"}\NormalTok{)[, }\StringTok{"yes"}\NormalTok{]}
\NormalTok{roc\_rf }\OtherTok{\textless{}{-}}\NormalTok{ pROC}\SpecialCharTok{::}\FunctionTok{roc}\NormalTok{(}\FunctionTok{as.numeric}\NormalTok{(test\_rf}\SpecialCharTok{$}\NormalTok{Outcome }\SpecialCharTok{==} \StringTok{"yes"}\NormalTok{), test\_rf}\SpecialCharTok{$}\NormalTok{pred\_rf)}
\end{Highlighting}
\end{Shaded}

\begin{verbatim}
## Setting levels: control = 0, case = 1
\end{verbatim}

\begin{verbatim}
## Setting direction: controls < cases
\end{verbatim}

\begin{Shaded}
\begin{Highlighting}[]
\NormalTok{pROC}\SpecialCharTok{::}\FunctionTok{auc}\NormalTok{(roc\_rf)}
\end{Highlighting}
\end{Shaded}

\begin{verbatim}
## Area under the curve: 0.8303
\end{verbatim}

\begin{Shaded}
\begin{Highlighting}[]
\FunctionTok{plot}\NormalTok{(roc\_rf, }\AttributeTok{main =} \FunctionTok{paste}\NormalTok{(}\StringTok{"Random Forest ROC (AUC ="}\NormalTok{, }\FunctionTok{round}\NormalTok{(pROC}\SpecialCharTok{::}\FunctionTok{auc}\NormalTok{(roc\_rf),}\DecValTok{3}\NormalTok{), }\StringTok{")"}\NormalTok{))}
\end{Highlighting}
\end{Shaded}

\includegraphics{Final-Projet_files/figure-latex/rf-eval-1.pdf}

\begin{Shaded}
\begin{Highlighting}[]
\CommentTok{\# Variable importance}
\NormalTok{vip}\SpecialCharTok{::}\FunctionTok{vip}\NormalTok{(rf\_fit}\SpecialCharTok{$}\NormalTok{finalModel, }\AttributeTok{num\_features =} \DecValTok{10}\NormalTok{) }\SpecialCharTok{+} \FunctionTok{ggtitle}\NormalTok{(}\StringTok{"Random Forest {-} Variable Importance"}\NormalTok{)}
\end{Highlighting}
\end{Shaded}

\includegraphics{Final-Projet_files/figure-latex/rf-eval-2.pdf}

Estimated output description: - caret reports cross-validated ROC for
different mtry values. - Test AUC likely similar or slightly higher than
logistic; variable importance shows Glucose near top, BMI, Age, and
DiabetesPedigreeFunction among others.

\subsection{XGBoost (gradient
boosting)}\label{xgboost-gradient-boosting}

Description: Train an XGBoost model (via caret or native xgboost) tuned
with repeated cross-validation. XGBoost often achieves the best AUC on
tabular datasets.

Expected output: - Tuned hyperparameters and CV performance table. -
Test set AUC and importance plot.

\begin{Shaded}
\begin{Highlighting}[]
\CommentTok{\# {-}{-}{-}{-} Robust XGBoost training chunk (fixes \textquotesingle{}one or more factor levels ...\textquotesingle{} error) {-}{-}{-}{-}}
\FunctionTok{library}\NormalTok{(caret)}
\FunctionTok{library}\NormalTok{(xgboost)}
\FunctionTok{set.seed}\NormalTok{(}\DecValTok{123}\NormalTok{)}

\CommentTok{\# Helper: robust mapping of Outcome {-}\textgreater{} "no"/"yes"}
\NormalTok{map\_outcome\_yesno }\OtherTok{\textless{}{-}} \ControlFlowTok{function}\NormalTok{(x) \{}
\NormalTok{  x\_chr }\OtherTok{\textless{}{-}} \FunctionTok{as.character}\NormalTok{(x)}
  \CommentTok{\# try numeric conversion}
\NormalTok{  x\_num }\OtherTok{\textless{}{-}} \FunctionTok{suppressWarnings}\NormalTok{(}\FunctionTok{as.numeric}\NormalTok{(x\_chr))}
  \ControlFlowTok{if}\NormalTok{ (}\FunctionTok{all}\NormalTok{(}\SpecialCharTok{!}\FunctionTok{is.na}\NormalTok{(x\_num))) \{}
    \FunctionTok{return}\NormalTok{(}\FunctionTok{ifelse}\NormalTok{(x\_num }\SpecialCharTok{==} \DecValTok{1}\NormalTok{, }\StringTok{"yes"}\NormalTok{, }\StringTok{"no"}\NormalTok{))}
\NormalTok{  \} }\ControlFlowTok{else}\NormalTok{ \{}
    \CommentTok{\# fallback: map common textual variants}
\NormalTok{    x\_low }\OtherTok{\textless{}{-}} \FunctionTok{tolower}\NormalTok{(x\_chr)}
    \FunctionTok{return}\NormalTok{(}\FunctionTok{ifelse}\NormalTok{(x\_low }\SpecialCharTok{\%in\%} \FunctionTok{c}\NormalTok{(}\StringTok{"1"}\NormalTok{,}\StringTok{"yes"}\NormalTok{,}\StringTok{"y"}\NormalTok{,}\StringTok{"true"}\NormalTok{,}\StringTok{"t"}\NormalTok{,}\StringTok{"positive"}\NormalTok{,}\StringTok{"pos"}\NormalTok{), }\StringTok{"yes"}\NormalTok{, }\StringTok{"no"}\NormalTok{))}
\NormalTok{  \}}
\NormalTok{\}}

\CommentTok{\# Ensure we have diabetes\_imputed available (the cleaned/imputed dataset)}
\ControlFlowTok{if}\NormalTok{ (}\SpecialCharTok{!}\FunctionTok{exists}\NormalTok{(}\StringTok{"diabetes\_imputed"}\NormalTok{)) }\FunctionTok{stop}\NormalTok{(}\StringTok{"diabetes\_imputed not found. Run data cleaning \& imputation first."}\NormalTok{)}

\CommentTok{\# Ensure we have a properly stratified train/test split with both classes present in train}
\NormalTok{create\_stratified\_split }\OtherTok{\textless{}{-}} \ControlFlowTok{function}\NormalTok{(data, }\AttributeTok{p =} \FloatTok{0.75}\NormalTok{, }\AttributeTok{times =} \DecValTok{10}\NormalTok{, }\AttributeTok{outcome\_col =} \StringTok{"Outcome"}\NormalTok{) \{}
  \ControlFlowTok{for}\NormalTok{ (i }\ControlFlowTok{in} \FunctionTok{seq\_len}\NormalTok{(times)) \{}
\NormalTok{    idx }\OtherTok{\textless{}{-}}\NormalTok{ caret}\SpecialCharTok{::}\FunctionTok{createDataPartition}\NormalTok{(data[[outcome\_col]], }\AttributeTok{p =}\NormalTok{ p, }\AttributeTok{list =} \ConstantTok{FALSE}\NormalTok{)}
\NormalTok{    tr }\OtherTok{\textless{}{-}}\NormalTok{ data[idx, , drop }\OtherTok{=} \ConstantTok{FALSE}\NormalTok{]}
\NormalTok{    te }\OtherTok{\textless{}{-}}\NormalTok{ data[}\SpecialCharTok{{-}}\NormalTok{idx, , drop }\OtherTok{=} \ConstantTok{FALSE}\NormalTok{]}
    \CommentTok{\# Map to yes/no to count reliably (without modifying original objects)}
\NormalTok{    tr\_map }\OtherTok{\textless{}{-}} \FunctionTok{table}\NormalTok{(}\FunctionTok{map\_outcome\_yesno}\NormalTok{(tr[[outcome\_col]]))}
    \ControlFlowTok{if}\NormalTok{ (}\FunctionTok{length}\NormalTok{(tr\_map) }\SpecialCharTok{==} \DecValTok{2}\NormalTok{) \{}
      \FunctionTok{return}\NormalTok{(}\FunctionTok{list}\NormalTok{(}\AttributeTok{train =}\NormalTok{ tr, }\AttributeTok{test =}\NormalTok{ te))}
\NormalTok{    \}}
\NormalTok{  \}}
  \FunctionTok{stop}\NormalTok{(}\StringTok{"Unable to create a stratified split that contains both Outcome classes in the training set. Check your data."}\NormalTok{)}
\NormalTok{\}}

\CommentTok{\# If train/test already exist check their class counts, otherwise create them}
\ControlFlowTok{if}\NormalTok{ (}\FunctionTok{exists}\NormalTok{(}\StringTok{"train"}\NormalTok{) }\SpecialCharTok{\&\&} \FunctionTok{exists}\NormalTok{(}\StringTok{"test"}\NormalTok{)) \{}
\NormalTok{  train\_counts }\OtherTok{\textless{}{-}} \FunctionTok{table}\NormalTok{(}\FunctionTok{map\_outcome\_yesno}\NormalTok{(train}\SpecialCharTok{$}\NormalTok{Outcome))}
\NormalTok{  test\_counts  }\OtherTok{\textless{}{-}} \FunctionTok{table}\NormalTok{(}\FunctionTok{map\_outcome\_yesno}\NormalTok{(test}\SpecialCharTok{$}\NormalTok{Outcome))}
  \FunctionTok{message}\NormalTok{(}\StringTok{"Existing split class counts (train): "}\NormalTok{, }\FunctionTok{paste}\NormalTok{(}\FunctionTok{names}\NormalTok{(train\_counts), train\_counts, }\AttributeTok{sep=}\StringTok{":"}\NormalTok{, }\AttributeTok{collapse =} \StringTok{" | "}\NormalTok{))}
  \FunctionTok{message}\NormalTok{(}\StringTok{"Existing split class counts (test) : "}\NormalTok{, }\FunctionTok{paste}\NormalTok{(}\FunctionTok{names}\NormalTok{(test\_counts), test\_counts, }\AttributeTok{sep=}\StringTok{":"}\NormalTok{, }\AttributeTok{collapse =} \StringTok{" | "}\NormalTok{))}
  \CommentTok{\# If training set lacks one class, recreate split}
  \ControlFlowTok{if}\NormalTok{ (}\FunctionTok{length}\NormalTok{(train\_counts) }\SpecialCharTok{\textless{}} \DecValTok{2}\NormalTok{) \{}
    \FunctionTok{message}\NormalTok{(}\StringTok{"Training set lacks one class {-}\textgreater{} recreating stratified split from diabetes\_imputed"}\NormalTok{)}
\NormalTok{    splits }\OtherTok{\textless{}{-}} \FunctionTok{create\_stratified\_split}\NormalTok{(diabetes\_imputed, }\AttributeTok{p =} \FloatTok{0.75}\NormalTok{, }\AttributeTok{times =} \DecValTok{20}\NormalTok{, }\AttributeTok{outcome\_col =} \StringTok{"Outcome"}\NormalTok{)}
\NormalTok{    train }\OtherTok{\textless{}{-}}\NormalTok{ splits}\SpecialCharTok{$}\NormalTok{train; test }\OtherTok{\textless{}{-}}\NormalTok{ splits}\SpecialCharTok{$}\NormalTok{test}
\NormalTok{  \}}
\NormalTok{\} }\ControlFlowTok{else}\NormalTok{ \{}
  \CommentTok{\# create split}
  \FunctionTok{message}\NormalTok{(}\StringTok{"No existing train/test found {-}\textgreater{} creating stratified split from diabetes\_imputed"}\NormalTok{)}
\NormalTok{  splits }\OtherTok{\textless{}{-}} \FunctionTok{create\_stratified\_split}\NormalTok{(diabetes\_imputed, }\AttributeTok{p =} \FloatTok{0.75}\NormalTok{, }\AttributeTok{times =} \DecValTok{20}\NormalTok{, }\AttributeTok{outcome\_col =} \StringTok{"Outcome"}\NormalTok{)}
\NormalTok{  train }\OtherTok{\textless{}{-}}\NormalTok{ splits}\SpecialCharTok{$}\NormalTok{train; test }\OtherTok{\textless{}{-}}\NormalTok{ splits}\SpecialCharTok{$}\NormalTok{test}
\NormalTok{\}}

\CommentTok{\# Now coerce train and test Outcome to factor with levels c("no","yes") using robust mapping}
\NormalTok{train\_rf }\OtherTok{\textless{}{-}}\NormalTok{ train }\SpecialCharTok{\%\textgreater{}\%} \FunctionTok{mutate}\NormalTok{(}\AttributeTok{Outcome =} \FunctionTok{factor}\NormalTok{(}\FunctionTok{map\_outcome\_yesno}\NormalTok{(Outcome), }\AttributeTok{levels =} \FunctionTok{c}\NormalTok{(}\StringTok{"no"}\NormalTok{, }\StringTok{"yes"}\NormalTok{)))}
\NormalTok{test\_rf  }\OtherTok{\textless{}{-}}\NormalTok{ test  }\SpecialCharTok{\%\textgreater{}\%} \FunctionTok{mutate}\NormalTok{(}\AttributeTok{Outcome =} \FunctionTok{factor}\NormalTok{(}\FunctionTok{map\_outcome\_yesno}\NormalTok{(Outcome), }\AttributeTok{levels =} \FunctionTok{c}\NormalTok{(}\StringTok{"no"}\NormalTok{, }\StringTok{"yes"}\NormalTok{)))}

\CommentTok{\# Final sanity check: both classes present in train}
\NormalTok{train\_table }\OtherTok{\textless{}{-}} \FunctionTok{table}\NormalTok{(train\_rf}\SpecialCharTok{$}\NormalTok{Outcome)}
\ControlFlowTok{if}\NormalTok{ (}\FunctionTok{length}\NormalTok{(train\_table) }\SpecialCharTok{\textless{}} \DecValTok{2}\NormalTok{) \{}
  \FunctionTok{stop}\NormalTok{(}\StringTok{"After mapping, training set still lacks one class. Aborting. Table: "}\NormalTok{, }\FunctionTok{paste}\NormalTok{(}\FunctionTok{names}\NormalTok{(train\_table), train\_table, }\AttributeTok{collapse =} \StringTok{"; "}\NormalTok{))}
\NormalTok{\}}
\FunctionTok{message}\NormalTok{(}\StringTok{"Final training class counts: "}\NormalTok{, }\FunctionTok{paste}\NormalTok{(}\FunctionTok{names}\NormalTok{(train\_table), train\_table, }\AttributeTok{sep=}\StringTok{":"}\NormalTok{, }\AttributeTok{collapse =} \StringTok{" | "}\NormalTok{))}

\CommentTok{\# {-}{-}{-}{-} trainControl: 3{-}fold CV, optimize for ROC {-}{-}{-}{-}}
\NormalTok{fast\_ctrl }\OtherTok{\textless{}{-}} \FunctionTok{trainControl}\NormalTok{(}
  \AttributeTok{method =} \StringTok{"cv"}\NormalTok{,}
  \AttributeTok{number =} \DecValTok{3}\NormalTok{,}
  \AttributeTok{classProbs =} \ConstantTok{TRUE}\NormalTok{,}
  \AttributeTok{summaryFunction =}\NormalTok{ twoClassSummary,}
  \AttributeTok{savePredictions =} \StringTok{"final"}\NormalTok{,}
  \AttributeTok{allowParallel =} \ConstantTok{TRUE}
\NormalTok{)}

\CommentTok{\# {-}{-}{-}{-} small tuning grid for xgbTree {-}{-}{-}{-}}
\NormalTok{xgb\_grid }\OtherTok{\textless{}{-}} \FunctionTok{expand.grid}\NormalTok{(}
  \AttributeTok{nrounds =} \FunctionTok{c}\NormalTok{(}\DecValTok{50}\NormalTok{, }\DecValTok{100}\NormalTok{),}
  \AttributeTok{max\_depth =} \FunctionTok{c}\NormalTok{(}\DecValTok{3}\NormalTok{, }\DecValTok{5}\NormalTok{),}
  \AttributeTok{eta =} \FunctionTok{c}\NormalTok{(}\FloatTok{0.1}\NormalTok{, }\FloatTok{0.3}\NormalTok{),}
  \AttributeTok{gamma =} \FunctionTok{c}\NormalTok{(}\DecValTok{0}\NormalTok{, }\DecValTok{1}\NormalTok{),}
  \AttributeTok{colsample\_bytree =} \FunctionTok{c}\NormalTok{(}\FloatTok{0.8}\NormalTok{),}
  \AttributeTok{min\_child\_weight =} \FunctionTok{c}\NormalTok{(}\DecValTok{1}\NormalTok{),}
  \AttributeTok{subsample =} \FunctionTok{c}\NormalTok{(}\FloatTok{0.8}\NormalTok{)}
\NormalTok{)}

\CommentTok{\# Train}
\FunctionTok{set.seed}\NormalTok{(}\DecValTok{123}\NormalTok{)}
\NormalTok{xgb\_fit\_fast }\OtherTok{\textless{}{-}} \FunctionTok{train}\NormalTok{(}
\NormalTok{  Outcome }\SpecialCharTok{\textasciitilde{}}\NormalTok{ Pregnancies }\SpecialCharTok{+}\NormalTok{ Glucose }\SpecialCharTok{+}\NormalTok{ BloodPressure }\SpecialCharTok{+}\NormalTok{ SkinThickness }\SpecialCharTok{+}
\NormalTok{    Insulin }\SpecialCharTok{+}\NormalTok{ BMI }\SpecialCharTok{+}\NormalTok{ DiabetesPedigreeFunction }\SpecialCharTok{+}\NormalTok{ Age,}
  \AttributeTok{data =}\NormalTok{ train\_rf,}
  \AttributeTok{method =} \StringTok{"xgbTree"}\NormalTok{,}
  \AttributeTok{metric =} \StringTok{"ROC"}\NormalTok{,}
  \AttributeTok{trControl =}\NormalTok{ fast\_ctrl,}
  \AttributeTok{tuneGrid =}\NormalTok{ xgb\_grid,}
  \AttributeTok{verbose =} \ConstantTok{FALSE}
\NormalTok{)}
\end{Highlighting}
\end{Shaded}

\begin{verbatim}
## [23:22:02] WARNING: src/c_api/c_api.cc:935: `ntree_limit` is deprecated, use `iteration_range` instead.
## [23:22:03] WARNING: src/c_api/c_api.cc:935: `ntree_limit` is deprecated, use `iteration_range` instead.
## [23:22:04] WARNING: src/c_api/c_api.cc:935: `ntree_limit` is deprecated, use `iteration_range` instead.
## [23:22:04] WARNING: src/c_api/c_api.cc:935: `ntree_limit` is deprecated, use `iteration_range` instead.
## [23:22:05] WARNING: src/c_api/c_api.cc:935: `ntree_limit` is deprecated, use `iteration_range` instead.
## [23:22:05] WARNING: src/c_api/c_api.cc:935: `ntree_limit` is deprecated, use `iteration_range` instead.
## [23:22:06] WARNING: src/c_api/c_api.cc:935: `ntree_limit` is deprecated, use `iteration_range` instead.
## [23:22:06] WARNING: src/c_api/c_api.cc:935: `ntree_limit` is deprecated, use `iteration_range` instead.
## [23:22:06] WARNING: src/c_api/c_api.cc:935: `ntree_limit` is deprecated, use `iteration_range` instead.
## [23:22:06] WARNING: src/c_api/c_api.cc:935: `ntree_limit` is deprecated, use `iteration_range` instead.
## [23:22:07] WARNING: src/c_api/c_api.cc:935: `ntree_limit` is deprecated, use `iteration_range` instead.
## [23:22:07] WARNING: src/c_api/c_api.cc:935: `ntree_limit` is deprecated, use `iteration_range` instead.
## [23:22:08] WARNING: src/c_api/c_api.cc:935: `ntree_limit` is deprecated, use `iteration_range` instead.
## [23:22:08] WARNING: src/c_api/c_api.cc:935: `ntree_limit` is deprecated, use `iteration_range` instead.
## [23:22:08] WARNING: src/c_api/c_api.cc:935: `ntree_limit` is deprecated, use `iteration_range` instead.
## [23:22:08] WARNING: src/c_api/c_api.cc:935: `ntree_limit` is deprecated, use `iteration_range` instead.
## [23:22:09] WARNING: src/c_api/c_api.cc:935: `ntree_limit` is deprecated, use `iteration_range` instead.
## [23:22:09] WARNING: src/c_api/c_api.cc:935: `ntree_limit` is deprecated, use `iteration_range` instead.
## [23:22:10] WARNING: src/c_api/c_api.cc:935: `ntree_limit` is deprecated, use `iteration_range` instead.
## [23:22:10] WARNING: src/c_api/c_api.cc:935: `ntree_limit` is deprecated, use `iteration_range` instead.
## [23:22:11] WARNING: src/c_api/c_api.cc:935: `ntree_limit` is deprecated, use `iteration_range` instead.
## [23:22:11] WARNING: src/c_api/c_api.cc:935: `ntree_limit` is deprecated, use `iteration_range` instead.
## [23:22:11] WARNING: src/c_api/c_api.cc:935: `ntree_limit` is deprecated, use `iteration_range` instead.
## [23:22:11] WARNING: src/c_api/c_api.cc:935: `ntree_limit` is deprecated, use `iteration_range` instead.
## [23:22:11] WARNING: src/c_api/c_api.cc:935: `ntree_limit` is deprecated, use `iteration_range` instead.
## [23:22:11] WARNING: src/c_api/c_api.cc:935: `ntree_limit` is deprecated, use `iteration_range` instead.
## [23:22:12] WARNING: src/c_api/c_api.cc:935: `ntree_limit` is deprecated, use `iteration_range` instead.
## [23:22:12] WARNING: src/c_api/c_api.cc:935: `ntree_limit` is deprecated, use `iteration_range` instead.
## [23:22:12] WARNING: src/c_api/c_api.cc:935: `ntree_limit` is deprecated, use `iteration_range` instead.
## [23:22:12] WARNING: src/c_api/c_api.cc:935: `ntree_limit` is deprecated, use `iteration_range` instead.
## [23:22:12] WARNING: src/c_api/c_api.cc:935: `ntree_limit` is deprecated, use `iteration_range` instead.
## [23:22:12] WARNING: src/c_api/c_api.cc:935: `ntree_limit` is deprecated, use `iteration_range` instead.
## [23:22:13] WARNING: src/c_api/c_api.cc:935: `ntree_limit` is deprecated, use `iteration_range` instead.
## [23:22:13] WARNING: src/c_api/c_api.cc:935: `ntree_limit` is deprecated, use `iteration_range` instead.
## [23:22:13] WARNING: src/c_api/c_api.cc:935: `ntree_limit` is deprecated, use `iteration_range` instead.
## [23:22:13] WARNING: src/c_api/c_api.cc:935: `ntree_limit` is deprecated, use `iteration_range` instead.
## [23:22:13] WARNING: src/c_api/c_api.cc:935: `ntree_limit` is deprecated, use `iteration_range` instead.
## [23:22:13] WARNING: src/c_api/c_api.cc:935: `ntree_limit` is deprecated, use `iteration_range` instead.
## [23:22:14] WARNING: src/c_api/c_api.cc:935: `ntree_limit` is deprecated, use `iteration_range` instead.
## [23:22:14] WARNING: src/c_api/c_api.cc:935: `ntree_limit` is deprecated, use `iteration_range` instead.
## [23:22:14] WARNING: src/c_api/c_api.cc:935: `ntree_limit` is deprecated, use `iteration_range` instead.
## [23:22:14] WARNING: src/c_api/c_api.cc:935: `ntree_limit` is deprecated, use `iteration_range` instead.
## [23:22:14] WARNING: src/c_api/c_api.cc:935: `ntree_limit` is deprecated, use `iteration_range` instead.
## [23:22:14] WARNING: src/c_api/c_api.cc:935: `ntree_limit` is deprecated, use `iteration_range` instead.
## [23:22:14] WARNING: src/c_api/c_api.cc:935: `ntree_limit` is deprecated, use `iteration_range` instead.
## [23:22:14] WARNING: src/c_api/c_api.cc:935: `ntree_limit` is deprecated, use `iteration_range` instead.
## [23:22:15] WARNING: src/c_api/c_api.cc:935: `ntree_limit` is deprecated, use `iteration_range` instead.
## [23:22:15] WARNING: src/c_api/c_api.cc:935: `ntree_limit` is deprecated, use `iteration_range` instead.
\end{verbatim}

\begin{Shaded}
\begin{Highlighting}[]
\CommentTok{\# Print results}
\NormalTok{xgb\_fit\_fast}
\end{Highlighting}
\end{Shaded}

\begin{verbatim}
## eXtreme Gradient Boosting 
## 
## 576 samples
##   8 predictor
##   2 classes: 'no', 'yes' 
## 
## No pre-processing
## Resampling: Cross-Validated (3 fold) 
## Summary of sample sizes: 384, 384, 384 
## Resampling results across tuning parameters:
## 
##   eta  max_depth  gamma  nrounds  ROC        Sens       Spec     
##   0.1  3          0       50      0.8390448  0.8346667  0.6368159
##   0.1  3          0      100      0.8275423  0.8320000  0.6368159
##   0.1  3          1       50      0.8294527  0.8400000  0.6119403
##   0.1  3          1      100      0.8188657  0.8106667  0.6318408
##   0.1  5          0       50      0.8214129  0.8240000  0.6218905
##   0.1  5          0      100      0.8119403  0.8160000  0.6318408
##   0.1  5          1       50      0.8263881  0.8213333  0.6218905
##   0.1  5          1      100      0.8119403  0.8000000  0.6119403
##   0.3  3          0       50      0.8014328  0.8186667  0.6019900
##   0.3  3          0      100      0.7893731  0.7946667  0.6019900
##   0.3  3          1       50      0.8005572  0.8053333  0.6119403
##   0.3  3          1      100      0.7937512  0.7813333  0.6119403
##   0.3  5          0       50      0.7985672  0.7946667  0.6019900
##   0.3  5          0      100      0.7865871  0.7840000  0.5970149
##   0.3  5          1       50      0.8008358  0.8080000  0.5920398
##   0.3  5          1      100      0.7936716  0.7946667  0.5920398
## 
## Tuning parameter 'colsample_bytree' was held constant at a value of 0.8
## 
## Tuning parameter 'min_child_weight' was held constant at a value of 1
## 
## Tuning parameter 'subsample' was held constant at a value of 0.8
## ROC was used to select the optimal model using the largest value.
## The final values used for the model were nrounds = 50, max_depth = 3, eta
##  = 0.1, gamma = 0, colsample_bytree = 0.8, min_child_weight = 1 and subsample
##  = 0.8.
\end{verbatim}

Evaluation on test set

\begin{Shaded}
\begin{Highlighting}[]
\NormalTok{test\_rf}\SpecialCharTok{$}\NormalTok{pred\_xgb }\OtherTok{\textless{}{-}} \FunctionTok{predict}\NormalTok{(xgb\_fit\_fast, }\AttributeTok{newdata =}\NormalTok{ test\_rf, }\AttributeTok{type =} \StringTok{"prob"}\NormalTok{)[, }\StringTok{"yes"}\NormalTok{]}
\NormalTok{roc\_xgb }\OtherTok{\textless{}{-}}\NormalTok{ pROC}\SpecialCharTok{::}\FunctionTok{roc}\NormalTok{(}\FunctionTok{as.numeric}\NormalTok{(test\_rf}\SpecialCharTok{$}\NormalTok{Outcome }\SpecialCharTok{==} \StringTok{"yes"}\NormalTok{), test\_rf}\SpecialCharTok{$}\NormalTok{pred\_xgb)}
\end{Highlighting}
\end{Shaded}

\begin{verbatim}
## Setting levels: control = 0, case = 1
\end{verbatim}

\begin{verbatim}
## Setting direction: controls < cases
\end{verbatim}

\begin{Shaded}
\begin{Highlighting}[]
\NormalTok{pROC}\SpecialCharTok{::}\FunctionTok{auc}\NormalTok{(roc\_xgb)}
\end{Highlighting}
\end{Shaded}

\begin{verbatim}
## Area under the curve: 0.8318
\end{verbatim}

\begin{Shaded}
\begin{Highlighting}[]
\FunctionTok{plot}\NormalTok{(roc\_xgb, }\AttributeTok{main =} \FunctionTok{paste}\NormalTok{(}\StringTok{"XGBoost ROC (AUC ="}\NormalTok{, }\FunctionTok{round}\NormalTok{(pROC}\SpecialCharTok{::}\FunctionTok{auc}\NormalTok{(roc\_xgb),}\DecValTok{3}\NormalTok{), }\StringTok{")"}\NormalTok{))}
\end{Highlighting}
\end{Shaded}

\includegraphics{Final-Projet_files/figure-latex/xgb-eval-1.pdf}

\begin{Shaded}
\begin{Highlighting}[]
\NormalTok{vip}\SpecialCharTok{::}\FunctionTok{vip}\NormalTok{(xgb\_fit\_fast}\SpecialCharTok{$}\NormalTok{finalModel, }\AttributeTok{num\_features =} \DecValTok{10}\NormalTok{) }\SpecialCharTok{+} \FunctionTok{ggtitle}\NormalTok{(}\StringTok{"XGBoost {-} Variable Importance"}\NormalTok{)}
\end{Highlighting}
\end{Shaded}

\includegraphics{Final-Projet_files/figure-latex/xgb-eval-2.pdf}

Estimated output description: - xgb\_fit prints best hyperparameters
(eta, max\_depth, nrounds). - XGBoost AUC often slightly higher or
similar to RF. - Variable importance may highlight Glucose, Age, BMI,
and Insulin.

\subsection{Model comparison (ROC curves
combined)}\label{model-comparison-roc-curves-combined}

Description: Plot ROC curves from the three models together for easy
comparison and compute AUC table.

Expected output: - Combined ROC plot with AUCs in the legend. - Table of
model AUCs.

\begin{Shaded}
\begin{Highlighting}[]
\NormalTok{roc\_g }\OtherTok{\textless{}{-}} \FunctionTok{ggroc}\NormalTok{(}\FunctionTok{list}\NormalTok{(}\AttributeTok{Logistic =}\NormalTok{ roc\_glm, }\AttributeTok{RF =}\NormalTok{ roc\_rf, }\AttributeTok{XGBoost =}\NormalTok{ roc\_xgb), }\AttributeTok{legacy.axes =} \ConstantTok{TRUE}\NormalTok{) }\SpecialCharTok{+}
  \FunctionTok{ggtitle}\NormalTok{(}\StringTok{"Model ROC Comparison"}\NormalTok{) }\SpecialCharTok{+}
  \FunctionTok{theme\_minimal}\NormalTok{()}

\NormalTok{roc\_g}
\end{Highlighting}
\end{Shaded}

\includegraphics{Final-Projet_files/figure-latex/compare-rocs-1.pdf}

\begin{Shaded}
\begin{Highlighting}[]
\CommentTok{\# AUC table}
\NormalTok{auc\_table }\OtherTok{\textless{}{-}} \FunctionTok{tibble}\NormalTok{(}
  \AttributeTok{model =} \FunctionTok{c}\NormalTok{(}\StringTok{"Logistic"}\NormalTok{, }\StringTok{"RandomForest"}\NormalTok{, }\StringTok{"XGBoost"}\NormalTok{),}
  \AttributeTok{AUC =} \FunctionTok{c}\NormalTok{(}\FunctionTok{as.numeric}\NormalTok{(pROC}\SpecialCharTok{::}\FunctionTok{auc}\NormalTok{(roc\_glm)), }\FunctionTok{as.numeric}\NormalTok{(pROC}\SpecialCharTok{::}\FunctionTok{auc}\NormalTok{(roc\_rf)), }\FunctionTok{as.numeric}\NormalTok{(pROC}\SpecialCharTok{::}\FunctionTok{auc}\NormalTok{(roc\_xgb)))}
\NormalTok{)}
\NormalTok{auc\_table }\SpecialCharTok{\%\textgreater{}\%} \FunctionTok{kable}\NormalTok{() }\SpecialCharTok{\%\textgreater{}\%} \FunctionTok{kable\_styling}\NormalTok{(}\AttributeTok{full\_width =} \ConstantTok{FALSE}\NormalTok{)}
\end{Highlighting}
\end{Shaded}

\begin{longtable}[t]{lr}
\toprule
model & AUC\\
\midrule
Logistic & 0.8242388\\
RandomForest & 0.8302687\\
XGBoost & 0.8317612\\
\bottomrule
\end{longtable}

\subsection{Calibration plot}\label{calibration-plot}

Description: Calibration shows whether predicted probabilities
correspond to observed frequencies. Good calibration is important for
clinical decisions.

Expected output: - Calibration plot for the best model showing predicted
probability bins vs observed event rates. Ideally points close to the
diagonal.

\begin{Shaded}
\begin{Highlighting}[]
\CommentTok{\# Logistic regression predictions (if not already)}
\ControlFlowTok{if}\NormalTok{(}\SpecialCharTok{!}\StringTok{"pred\_glm"} \SpecialCharTok{\%in\%} \FunctionTok{colnames}\NormalTok{(test)) \{}
\NormalTok{  test}\SpecialCharTok{$}\NormalTok{pred\_glm }\OtherTok{\textless{}{-}} \FunctionTok{predict}\NormalTok{(glm\_fit, }\AttributeTok{newdata =}\NormalTok{ test, }\AttributeTok{type =} \StringTok{"response"}\NormalTok{)}
\NormalTok{\}}

\CommentTok{\# Random Forest predictions (if not already)}
\ControlFlowTok{if}\NormalTok{(}\SpecialCharTok{!}\StringTok{"pred\_rf"} \SpecialCharTok{\%in\%} \FunctionTok{colnames}\NormalTok{(test\_rf)) \{}
\NormalTok{  test\_rf}\SpecialCharTok{$}\NormalTok{pred\_rf }\OtherTok{\textless{}{-}} \FunctionTok{predict}\NormalTok{(rf\_fit, }\AttributeTok{newdata =}\NormalTok{ test\_rf, }\AttributeTok{type =} \StringTok{"prob"}\NormalTok{)[, }\StringTok{"yes"}\NormalTok{]}
\NormalTok{\}}

\CommentTok{\# XGBoost predictions (if caret XGBoost succeeded)}
\ControlFlowTok{if}\NormalTok{(}\FunctionTok{exists}\NormalTok{(}\StringTok{"xgb\_fit\_fast"}\NormalTok{) }\SpecialCharTok{\&\&} \SpecialCharTok{!}\StringTok{"pred\_xgb"} \SpecialCharTok{\%in\%} \FunctionTok{colnames}\NormalTok{(test\_rf)) \{}
\NormalTok{  test\_rf}\SpecialCharTok{$}\NormalTok{pred\_xgb }\OtherTok{\textless{}{-}} \FunctionTok{predict}\NormalTok{(xgb\_fit\_fast, }\AttributeTok{newdata =}\NormalTok{ test\_rf, }\AttributeTok{type =} \StringTok{"prob"}\NormalTok{)[, }\StringTok{"yes"}\NormalTok{]}
\NormalTok{\}}

\CommentTok{\# Make sure AUC table exists}
\ControlFlowTok{if}\NormalTok{(}\SpecialCharTok{!}\FunctionTok{exists}\NormalTok{(}\StringTok{"auc\_table"}\NormalTok{)) \{}
\NormalTok{  auc\_table }\OtherTok{\textless{}{-}} \FunctionTok{tibble}\NormalTok{(}
    \AttributeTok{model =} \FunctionTok{c}\NormalTok{(}\StringTok{"Logistic"}\NormalTok{, }\StringTok{"RandomForest"}\NormalTok{, }\StringTok{"XGBoost"}\NormalTok{),}
    \AttributeTok{AUC =} \FunctionTok{c}\NormalTok{(}\FunctionTok{as.numeric}\NormalTok{(pROC}\SpecialCharTok{::}\FunctionTok{auc}\NormalTok{(roc\_glm)),}
            \FunctionTok{as.numeric}\NormalTok{(pROC}\SpecialCharTok{::}\FunctionTok{auc}\NormalTok{(roc\_rf)),}
            \ControlFlowTok{if}\NormalTok{(}\FunctionTok{exists}\NormalTok{(}\StringTok{"roc\_xgb"}\NormalTok{)) }\FunctionTok{as.numeric}\NormalTok{(pROC}\SpecialCharTok{::}\FunctionTok{auc}\NormalTok{(roc\_xgb)) }\ControlFlowTok{else} \ConstantTok{NA}\NormalTok{)}
\NormalTok{  )}
\NormalTok{\}}

\CommentTok{\# Choose best model safely}
\NormalTok{best\_model\_name }\OtherTok{\textless{}{-}}\NormalTok{ auc\_table}\SpecialCharTok{$}\NormalTok{model[}\FunctionTok{which.max}\NormalTok{(auc\_table}\SpecialCharTok{$}\NormalTok{AUC)]}

\NormalTok{best\_pred }\OtherTok{\textless{}{-}} \ControlFlowTok{switch}\NormalTok{(best\_model\_name,}
                    \StringTok{"Logistic"} \OtherTok{=}\NormalTok{ test}\SpecialCharTok{$}\NormalTok{pred\_glm,}
                    \StringTok{"RandomForest"} \OtherTok{=}\NormalTok{ test\_rf}\SpecialCharTok{$}\NormalTok{pred\_rf,}
                    \StringTok{"XGBoost"} \OtherTok{=}\NormalTok{ \{}
                      \ControlFlowTok{if}\NormalTok{(}\StringTok{"pred\_xgb"} \SpecialCharTok{\%in\%} \FunctionTok{colnames}\NormalTok{(test\_rf)) test\_rf}\SpecialCharTok{$}\NormalTok{pred\_xgb }\ControlFlowTok{else}\NormalTok{ test\_rf}\SpecialCharTok{$}\NormalTok{pred\_rf}
\NormalTok{                    \})}
\end{Highlighting}
\end{Shaded}

\begin{Shaded}
\begin{Highlighting}[]
\FunctionTok{library}\NormalTok{(ggplot2)}
\FunctionTok{library}\NormalTok{(dplyr)}

\NormalTok{cal\_df }\OtherTok{\textless{}{-}} \FunctionTok{tibble}\NormalTok{(}\AttributeTok{truth =} \FunctionTok{ifelse}\NormalTok{(test\_rf}\SpecialCharTok{$}\NormalTok{Outcome }\SpecialCharTok{==} \StringTok{"yes"}\NormalTok{, }\DecValTok{1}\NormalTok{, }\DecValTok{0}\NormalTok{), }\AttributeTok{pred =}\NormalTok{ best\_pred) }\SpecialCharTok{\%\textgreater{}\%}
  \FunctionTok{mutate}\NormalTok{(}\AttributeTok{pred\_bin =} \FunctionTok{ntile}\NormalTok{(pred, }\DecValTok{10}\NormalTok{)) }\SpecialCharTok{\%\textgreater{}\%}
  \FunctionTok{group\_by}\NormalTok{(pred\_bin) }\SpecialCharTok{\%\textgreater{}\%}
  \FunctionTok{summarise}\NormalTok{(}\AttributeTok{mean\_pred =} \FunctionTok{mean}\NormalTok{(pred), }\AttributeTok{obs\_rate =} \FunctionTok{mean}\NormalTok{(truth), }\AttributeTok{.groups =} \StringTok{"drop"}\NormalTok{)}

\FunctionTok{ggplot}\NormalTok{(cal\_df, }\FunctionTok{aes}\NormalTok{(}\AttributeTok{x =}\NormalTok{ mean\_pred, }\AttributeTok{y =}\NormalTok{ obs\_rate)) }\SpecialCharTok{+}
  \FunctionTok{geom\_point}\NormalTok{(}\AttributeTok{size =} \DecValTok{3}\NormalTok{, }\AttributeTok{color =} \StringTok{"blue"}\NormalTok{) }\SpecialCharTok{+}
  \FunctionTok{geom\_line}\NormalTok{(}\AttributeTok{color =} \StringTok{"blue"}\NormalTok{) }\SpecialCharTok{+}
  \FunctionTok{geom\_abline}\NormalTok{(}\AttributeTok{slope =} \DecValTok{1}\NormalTok{, }\AttributeTok{intercept =} \DecValTok{0}\NormalTok{, }\AttributeTok{linetype =} \StringTok{"dashed"}\NormalTok{, }\AttributeTok{color =} \StringTok{"gray"}\NormalTok{) }\SpecialCharTok{+}
  \FunctionTok{labs}\NormalTok{(}\AttributeTok{x =} \StringTok{"Mean predicted probability"}\NormalTok{, }\AttributeTok{y =} \StringTok{"Observed event rate"}\NormalTok{,}
       \AttributeTok{title =} \FunctionTok{paste}\NormalTok{(}\StringTok{"Calibration plot (binned) {-}"}\NormalTok{, best\_model\_name)) }\SpecialCharTok{+}
  \FunctionTok{theme\_minimal}\NormalTok{()}
\end{Highlighting}
\end{Shaded}

\includegraphics{Final-Projet_files/figure-latex/unnamed-chunk-1-1.pdf}

\subsection{Explainability: permutation importance and
SHAP}\label{explainability-permutation-importance-and-shap}

Description: We include features beyond course coverage: SHAP
explanations (via fastshap) for the chosen model and permutation
importance (vip/permutation). This helps satisfy the rubric item
requiring a feature we didn't cover in class.

Expected output: - A permutation importance plot and a SHAP summary plot
showing feature contributions. - Example of an individual explanation
for a test case (force or waterfall-like explanation).

Permutation importance (using vip)

\begin{Shaded}
\begin{Highlighting}[]
\CommentTok{\# Manual permutation importance (works around vi\_permute issues)}
\FunctionTok{library}\NormalTok{(dplyr)}
\FunctionTok{library}\NormalTok{(pROC)}
\FunctionTok{library}\NormalTok{(ggplot2)}
\FunctionTok{set.seed}\NormalTok{(}\DecValTok{123}\NormalTok{)}

\CommentTok{\# Preconditions}
\ControlFlowTok{if}\NormalTok{ (}\SpecialCharTok{!}\FunctionTok{exists}\NormalTok{(}\StringTok{"rf\_fit"}\NormalTok{)) }\FunctionTok{stop}\NormalTok{(}\StringTok{"rf\_fit (caret::train) not found. Run Random Forest training first."}\NormalTok{)}
\ControlFlowTok{if}\NormalTok{ (}\SpecialCharTok{!}\FunctionTok{exists}\NormalTok{(}\StringTok{"train\_rf"}\NormalTok{)) }\FunctionTok{stop}\NormalTok{(}\StringTok{"train\_rf not found. Ensure train\_rf exists and was used to train rf\_fit."}\NormalTok{)}

\CommentTok{\# Predictors used to train the model (adjust if different)}
\NormalTok{model\_vars\_full }\OtherTok{\textless{}{-}} \FunctionTok{c}\NormalTok{(}\StringTok{"Pregnancies"}\NormalTok{,}\StringTok{"Glucose"}\NormalTok{,}\StringTok{"BloodPressure"}\NormalTok{,}\StringTok{"SkinThickness"}\NormalTok{,}
                     \StringTok{"Insulin"}\NormalTok{,}\StringTok{"BMI"}\NormalTok{,}\StringTok{"DiabetesPedigreeFunction"}\NormalTok{,}\StringTok{"Age"}\NormalTok{)}

\CommentTok{\# Variables to compute permutation importance for (subset of model\_vars\_full)}
\NormalTok{plot\_vars\_interest }\OtherTok{\textless{}{-}} \FunctionTok{c}\NormalTok{(}\StringTok{"Glucose"}\NormalTok{,}\StringTok{"BMI"}\NormalTok{,}\StringTok{"Insulin"}\NormalTok{,}\StringTok{"BloodPressure"}\NormalTok{,}\StringTok{"SkinThickness"}\NormalTok{,}\StringTok{"DiabetesPedigreeFunction"}\NormalTok{)}

\CommentTok{\# Validate}
\NormalTok{missing\_cols }\OtherTok{\textless{}{-}} \FunctionTok{setdiff}\NormalTok{(model\_vars\_full, }\FunctionTok{names}\NormalTok{(train\_rf))}
\ControlFlowTok{if}\NormalTok{ (}\FunctionTok{length}\NormalTok{(missing\_cols) }\SpecialCharTok{\textgreater{}} \DecValTok{0}\NormalTok{) }\FunctionTok{stop}\NormalTok{(}\StringTok{"train\_rf missing model columns: "}\NormalTok{, }\FunctionTok{paste}\NormalTok{(missing\_cols, }\AttributeTok{collapse =} \StringTok{", "}\NormalTok{))}

\CommentTok{\# Prepare data \& baseline}
\NormalTok{X\_train\_full }\OtherTok{\textless{}{-}}\NormalTok{ train\_rf }\SpecialCharTok{\%\textgreater{}\%} \FunctionTok{select}\NormalTok{(}\FunctionTok{all\_of}\NormalTok{(model\_vars\_full))}
\NormalTok{y\_train\_num  }\OtherTok{\textless{}{-}} \FunctionTok{ifelse}\NormalTok{(}\FunctionTok{as.character}\NormalTok{(train\_rf}\SpecialCharTok{$}\NormalTok{Outcome) }\SpecialCharTok{==} \StringTok{"yes"}\NormalTok{, }\DecValTok{1}\NormalTok{, }\DecValTok{0}\NormalTok{)}

\CommentTok{\# Baseline predicted probabilities \& AUC}
\NormalTok{base\_probs }\OtherTok{\textless{}{-}} \FunctionTok{predict}\NormalTok{(rf\_fit, }\AttributeTok{newdata =}\NormalTok{ X\_train\_full, }\AttributeTok{type =} \StringTok{"prob"}\NormalTok{)[, }\StringTok{"yes"}\NormalTok{]}
\ControlFlowTok{if}\NormalTok{ (}\FunctionTok{length}\NormalTok{(base\_probs) }\SpecialCharTok{!=} \FunctionTok{nrow}\NormalTok{(X\_train\_full)) }\FunctionTok{stop}\NormalTok{(}\StringTok{"Baseline prediction length mismatch."}\NormalTok{)}
\NormalTok{base\_auc }\OtherTok{\textless{}{-}} \FunctionTok{as.numeric}\NormalTok{(pROC}\SpecialCharTok{::}\FunctionTok{auc}\NormalTok{(pROC}\SpecialCharTok{::}\FunctionTok{roc}\NormalTok{(y\_train\_num, base\_probs)))}
\FunctionTok{message}\NormalTok{(}\StringTok{"Baseline AUC (training data) = "}\NormalTok{, }\FunctionTok{round}\NormalTok{(base\_auc, }\DecValTok{4}\NormalTok{))}

\CommentTok{\# Permutation loop: for each variable, do nsim permutations and average the decrease in AUC}
\NormalTok{nsim }\OtherTok{\textless{}{-}} \DecValTok{20}   \CommentTok{\# set to 50{-}100 for final; use 10{-}20 for quick runs}
\NormalTok{results }\OtherTok{\textless{}{-}} \FunctionTok{tibble}\NormalTok{(}\AttributeTok{Variable =}\NormalTok{ plot\_vars\_interest, }\AttributeTok{MeanDecreaseAUC =} \ConstantTok{NA\_real\_}\NormalTok{, }\AttributeTok{SDAUC =} \ConstantTok{NA\_real\_}\NormalTok{, }\AttributeTok{nsim =}\NormalTok{ nsim)}

\ControlFlowTok{for}\NormalTok{ (v }\ControlFlowTok{in} \FunctionTok{seq\_along}\NormalTok{(plot\_vars\_interest)) \{}
\NormalTok{  varname }\OtherTok{\textless{}{-}}\NormalTok{ plot\_vars\_interest[v]}
\NormalTok{  decreases }\OtherTok{\textless{}{-}} \FunctionTok{numeric}\NormalTok{(nsim)}
  \ControlFlowTok{for}\NormalTok{ (i }\ControlFlowTok{in} \FunctionTok{seq\_len}\NormalTok{(nsim)) \{}
\NormalTok{    Xp }\OtherTok{\textless{}{-}}\NormalTok{ X\_train\_full}
\NormalTok{    Xp[[varname]] }\OtherTok{\textless{}{-}} \FunctionTok{sample}\NormalTok{(Xp[[varname]], }\AttributeTok{size =} \FunctionTok{nrow}\NormalTok{(Xp), }\AttributeTok{replace =} \ConstantTok{FALSE}\NormalTok{)  }\CommentTok{\# permutation}
    \CommentTok{\# safe predict with error handling}
\NormalTok{    preds\_try }\OtherTok{\textless{}{-}} \FunctionTok{try}\NormalTok{(}\FunctionTok{predict}\NormalTok{(rf\_fit, }\AttributeTok{newdata =}\NormalTok{ Xp, }\AttributeTok{type =} \StringTok{"prob"}\NormalTok{)[, }\StringTok{"yes"}\NormalTok{], }\AttributeTok{silent =} \ConstantTok{TRUE}\NormalTok{)}
    \ControlFlowTok{if}\NormalTok{ (}\FunctionTok{inherits}\NormalTok{(preds\_try, }\StringTok{"try{-}error"}\NormalTok{)) \{}
      \FunctionTok{stop}\NormalTok{(}\StringTok{"Predict failed for variable "}\NormalTok{, varname, }\StringTok{" on permutation "}\NormalTok{, i, }\StringTok{": "}\NormalTok{, }\FunctionTok{as.character}\NormalTok{(preds\_try))}
\NormalTok{    \}}
\NormalTok{    auc\_perm }\OtherTok{\textless{}{-}} \FunctionTok{as.numeric}\NormalTok{(pROC}\SpecialCharTok{::}\FunctionTok{auc}\NormalTok{(pROC}\SpecialCharTok{::}\FunctionTok{roc}\NormalTok{(y\_train\_num, preds\_try)))}
    \CommentTok{\# decrease (baseline {-} permuted)}
\NormalTok{    decreases[i] }\OtherTok{\textless{}{-}}\NormalTok{ base\_auc }\SpecialCharTok{{-}}\NormalTok{ auc\_perm}
\NormalTok{  \}}
\NormalTok{  results}\SpecialCharTok{$}\NormalTok{MeanDecreaseAUC[v] }\OtherTok{\textless{}{-}} \FunctionTok{mean}\NormalTok{(decreases, }\AttributeTok{na.rm =} \ConstantTok{TRUE}\NormalTok{)}
\NormalTok{  results}\SpecialCharTok{$}\NormalTok{SDAUC[v] }\OtherTok{\textless{}{-}} \FunctionTok{sd}\NormalTok{(decreases, }\AttributeTok{na.rm =} \ConstantTok{TRUE}\NormalTok{)}
\NormalTok{\}}

\CommentTok{\# Tidy \& rank}
\NormalTok{results }\OtherTok{\textless{}{-}}\NormalTok{ results }\SpecialCharTok{\%\textgreater{}\%} \FunctionTok{arrange}\NormalTok{(}\FunctionTok{desc}\NormalTok{(MeanDecreaseAUC))}
\NormalTok{results }\SpecialCharTok{\%\textgreater{}\%}\NormalTok{ knitr}\SpecialCharTok{::}\FunctionTok{kable}\NormalTok{(}\AttributeTok{digits =} \DecValTok{4}\NormalTok{, }\AttributeTok{caption =} \StringTok{"Manual permutation importance (mean decrease in AUC)"}\NormalTok{)}
\end{Highlighting}
\end{Shaded}

\begin{longtable}[]{@{}lrrr@{}}
\caption{Manual permutation importance (mean decrease in
AUC)}\tabularnewline
\toprule\noalign{}
Variable & MeanDecreaseAUC & SDAUC & nsim \\
\midrule\noalign{}
\endfirsthead
\toprule\noalign{}
Variable & MeanDecreaseAUC & SDAUC & nsim \\
\midrule\noalign{}
\endhead
\bottomrule\noalign{}
\endlastfoot
Glucose & 0.1194 & 0.0107 & 20 \\
BMI & 0.0418 & 0.0057 & 20 \\
DiabetesPedigreeFunction & 0.0106 & 0.0016 & 20 \\
Insulin & 0.0055 & 0.0012 & 20 \\
SkinThickness & 0.0012 & 0.0003 & 20 \\
BloodPressure & 0.0010 & 0.0003 & 20 \\
\end{longtable}

\begin{Shaded}
\begin{Highlighting}[]
\CommentTok{\# Plot}
\NormalTok{results }\SpecialCharTok{\%\textgreater{}\%}
  \FunctionTok{mutate}\NormalTok{(}\AttributeTok{Variable =} \FunctionTok{factor}\NormalTok{(Variable, }\AttributeTok{levels =} \FunctionTok{rev}\NormalTok{(Variable))) }\SpecialCharTok{\%\textgreater{}\%}
  \FunctionTok{ggplot}\NormalTok{(}\FunctionTok{aes}\NormalTok{(}\AttributeTok{x =}\NormalTok{ MeanDecreaseAUC, }\AttributeTok{y =}\NormalTok{ Variable)) }\SpecialCharTok{+}
  \FunctionTok{geom\_col}\NormalTok{(}\AttributeTok{fill =} \StringTok{"steelblue"}\NormalTok{) }\SpecialCharTok{+}
  \FunctionTok{geom\_errorbarh}\NormalTok{(}\FunctionTok{aes}\NormalTok{(}\AttributeTok{xmin =}\NormalTok{ MeanDecreaseAUC }\SpecialCharTok{{-}}\NormalTok{ SDAUC, }\AttributeTok{xmax =}\NormalTok{ MeanDecreaseAUC }\SpecialCharTok{+}\NormalTok{ SDAUC), }\AttributeTok{height =} \FloatTok{0.2}\NormalTok{) }\SpecialCharTok{+}
  \FunctionTok{labs}\NormalTok{(}\AttributeTok{title =} \FunctionTok{paste0}\NormalTok{(}\StringTok{"Permutation importance (RF) — mean decrease in AUC (nsim="}\NormalTok{, nsim, }\StringTok{")"}\NormalTok{),}
       \AttributeTok{x =} \StringTok{"Mean decrease in AUC (higher = more important)"}\NormalTok{, }\AttributeTok{y =} \ConstantTok{NULL}\NormalTok{) }\SpecialCharTok{+}
  \FunctionTok{theme\_minimal}\NormalTok{(}\AttributeTok{base\_size =} \DecValTok{12}\NormalTok{)}
\end{Highlighting}
\end{Shaded}

\includegraphics{Final-Projet_files/figure-latex/perm-imp-fixed2-1.pdf}

\begin{Shaded}
\begin{Highlighting}[]
\CommentTok{\# Save results for reproducibility}
\FunctionTok{dir.create}\NormalTok{(}\StringTok{"outputs"}\NormalTok{, }\AttributeTok{showWarnings =} \ConstantTok{FALSE}\NormalTok{)}
\FunctionTok{saveRDS}\NormalTok{(results, }\AttributeTok{file =} \StringTok{"outputs/perm\_importance\_manual\_rf\_auc.rds"}\NormalTok{)}
\end{Highlighting}
\end{Shaded}

SHAP explanations (fastshap)

\begin{Shaded}
\begin{Highlighting}[]
\FunctionTok{library}\NormalTok{(fastshap)}
\FunctionTok{library}\NormalTok{(ggplot2)}

\FunctionTok{set.seed}\NormalTok{(}\DecValTok{123}\NormalTok{)}

\CommentTok{\# Safe wrapper for caret RF object}
\NormalTok{pred\_wrapper }\OtherTok{\textless{}{-}} \ControlFlowTok{function}\NormalTok{(object, newdata) \{}
  \FunctionTok{predict}\NormalTok{(object, }\AttributeTok{newdata =}\NormalTok{ newdata, }\AttributeTok{type =} \StringTok{"prob"}\NormalTok{)[, }\StringTok{"yes"}\NormalTok{]}
\NormalTok{\}}

\CommentTok{\# Subset for SHAP: all model predictors}
\NormalTok{X\_train\_shap }\OtherTok{\textless{}{-}}\NormalTok{ X\_train\_full}

\CommentTok{\# Compute SHAP values (approximate, 50 Monte Carlo repetitions for speed; increase for final)}
\NormalTok{shap\_vals }\OtherTok{\textless{}{-}}\NormalTok{ fastshap}\SpecialCharTok{::}\FunctionTok{explain}\NormalTok{(}
  \AttributeTok{object =}\NormalTok{ rf\_fit,}
  \AttributeTok{X =}\NormalTok{ X\_train\_shap,}
  \AttributeTok{pred\_wrapper =}\NormalTok{ pred\_wrapper,}
  \AttributeTok{nsim =} \DecValTok{50}
\NormalTok{)}

\CommentTok{\# SHAP summary plot (mean absolute effect per feature)}
\NormalTok{shap\_long }\OtherTok{\textless{}{-}}\NormalTok{ shap\_vals }\SpecialCharTok{\%\textgreater{}\%} 
  \FunctionTok{as.data.frame}\NormalTok{() }\SpecialCharTok{\%\textgreater{}\%} 
\NormalTok{  tidyr}\SpecialCharTok{::}\FunctionTok{pivot\_longer}\NormalTok{(}\FunctionTok{everything}\NormalTok{(), }\AttributeTok{names\_to =} \StringTok{"Feature"}\NormalTok{, }\AttributeTok{values\_to =} \StringTok{"SHAP"}\NormalTok{)}
  
\NormalTok{shap\_summary }\OtherTok{\textless{}{-}}\NormalTok{ shap\_long }\SpecialCharTok{\%\textgreater{}\%}
  \FunctionTok{group\_by}\NormalTok{(Feature) }\SpecialCharTok{\%\textgreater{}\%}
  \FunctionTok{summarise}\NormalTok{(}\AttributeTok{MeanAbsSHAP =} \FunctionTok{mean}\NormalTok{(}\FunctionTok{abs}\NormalTok{(SHAP))) }\SpecialCharTok{\%\textgreater{}\%}
  \FunctionTok{arrange}\NormalTok{(}\FunctionTok{desc}\NormalTok{(MeanAbsSHAP))}

\CommentTok{\# Plot}
\FunctionTok{ggplot}\NormalTok{(shap\_summary, }\FunctionTok{aes}\NormalTok{(}\AttributeTok{x =}\NormalTok{ MeanAbsSHAP, }\AttributeTok{y =} \FunctionTok{reorder}\NormalTok{(Feature, MeanAbsSHAP))) }\SpecialCharTok{+}
  \FunctionTok{geom\_col}\NormalTok{(}\AttributeTok{fill =} \StringTok{"darkorange"}\NormalTok{) }\SpecialCharTok{+}
  \FunctionTok{labs}\NormalTok{(}\AttributeTok{title =} \StringTok{"SHAP summary plot (mean absolute value)"}\NormalTok{, }\AttributeTok{x =} \StringTok{"Mean |SHAP|"}\NormalTok{, }\AttributeTok{y =} \ConstantTok{NULL}\NormalTok{) }\SpecialCharTok{+}
  \FunctionTok{theme\_minimal}\NormalTok{(}\AttributeTok{base\_size =} \DecValTok{12}\NormalTok{)}
\end{Highlighting}
\end{Shaded}

\includegraphics{Final-Projet_files/figure-latex/perm-imp-working-1.pdf}

Estimated output description: - Permutation importance ranking: Glucose,
BMI, Age, Insulin, DiabetesPedigreeFunction expected near top. - SHAP
summary: confirms global impacts and provides direction (positive SHAP
-\textgreater{} increases predicted probability).

Individual-level explanation example

\begin{Shaded}
\begin{Highlighting}[]
\CommentTok{\# SHAP (fastshap) with matching classes for X and newdata (forces data.frame)}
\FunctionTok{library}\NormalTok{(fastshap)}
\FunctionTok{library}\NormalTok{(dplyr)}
\FunctionTok{library}\NormalTok{(knitr)}
\FunctionTok{library}\NormalTok{(kableExtra)}
\FunctionTok{library}\NormalTok{(ggplot2)}

\FunctionTok{set.seed}\NormalTok{(}\DecValTok{123}\NormalTok{)}

\CommentTok{\# Preconditions}
\ControlFlowTok{if}\NormalTok{ (}\SpecialCharTok{!}\FunctionTok{exists}\NormalTok{(}\StringTok{"rf\_fit"}\NormalTok{)) }\FunctionTok{stop}\NormalTok{(}\StringTok{"rf\_fit (caret::train) not found. Run the Random Forest training chunk first."}\NormalTok{)}
\ControlFlowTok{if}\NormalTok{ (}\SpecialCharTok{!}\FunctionTok{exists}\NormalTok{(}\StringTok{"train\_rf"}\NormalTok{) }\SpecialCharTok{||} \SpecialCharTok{!}\FunctionTok{exists}\NormalTok{(}\StringTok{"test\_rf"}\NormalTok{)) }\FunctionTok{stop}\NormalTok{(}\StringTok{"train\_rf and/or test\_rf not found. Run the data split chunk first."}\NormalTok{)}

\CommentTok{\# Exact predictors used when training rf\_fit}
\NormalTok{model\_vars\_full }\OtherTok{\textless{}{-}} \FunctionTok{c}\NormalTok{(}\StringTok{"Pregnancies"}\NormalTok{,}\StringTok{"Glucose"}\NormalTok{,}\StringTok{"BloodPressure"}\NormalTok{,}\StringTok{"SkinThickness"}\NormalTok{,}
                     \StringTok{"Insulin"}\NormalTok{,}\StringTok{"BMI"}\NormalTok{,}\StringTok{"DiabetesPedigreeFunction"}\NormalTok{,}\StringTok{"Age"}\NormalTok{)}

\CommentTok{\# Coerce X\_train and newdata to the same class (data.frame)}
\NormalTok{X\_train\_full }\OtherTok{\textless{}{-}}\NormalTok{ train\_rf }\SpecialCharTok{\%\textgreater{}\%} \FunctionTok{select}\NormalTok{(}\FunctionTok{all\_of}\NormalTok{(model\_vars\_full)) }\SpecialCharTok{\%\textgreater{}\%} \FunctionTok{as.data.frame}\NormalTok{()}
\NormalTok{X\_test\_full  }\OtherTok{\textless{}{-}}\NormalTok{ test\_rf  }\SpecialCharTok{\%\textgreater{}\%} \FunctionTok{select}\NormalTok{(}\FunctionTok{all\_of}\NormalTok{(model\_vars\_full)) }\SpecialCharTok{\%\textgreater{}\%} \FunctionTok{as.data.frame}\NormalTok{()}

\CommentTok{\# Safe prediction wrapper returning probability for class "yes"}
\NormalTok{pred\_wrapper\_prob }\OtherTok{\textless{}{-}} \ControlFlowTok{function}\NormalTok{(object, newdata) \{}
  \CommentTok{\# Ensure newdata is a data.frame (caret predict handles tibbles too, but be consistent)}
\NormalTok{  newdata }\OtherTok{\textless{}{-}} \FunctionTok{as.data.frame}\NormalTok{(newdata)}
\NormalTok{  probs }\OtherTok{\textless{}{-}} \FunctionTok{predict}\NormalTok{(object, }\AttributeTok{newdata =}\NormalTok{ newdata, }\AttributeTok{type =} \StringTok{"prob"}\NormalTok{)}
  \ControlFlowTok{if}\NormalTok{ (}\SpecialCharTok{!}\NormalTok{(}\StringTok{"yes"} \SpecialCharTok{\%in\%} \FunctionTok{colnames}\NormalTok{(probs))) \{}
    \FunctionTok{stop}\NormalTok{(}\StringTok{"pred\_wrapper\_prob: predicted probabilities do not include column \textquotesingle{}yes\textquotesingle{}. Columns: "}\NormalTok{,}
         \FunctionTok{paste}\NormalTok{(}\FunctionTok{colnames}\NormalTok{(probs), }\AttributeTok{collapse =} \StringTok{", "}\NormalTok{))}
\NormalTok{  \}}
\NormalTok{  prob\_yes }\OtherTok{\textless{}{-}} \FunctionTok{as.numeric}\NormalTok{(probs[, }\StringTok{"yes"}\NormalTok{])}
  \ControlFlowTok{if}\NormalTok{ (}\FunctionTok{length}\NormalTok{(prob\_yes) }\SpecialCharTok{!=} \FunctionTok{nrow}\NormalTok{(newdata)) }\FunctionTok{stop}\NormalTok{(}\StringTok{"pred\_wrapper\_prob: length(prob\_yes) != nrow(newdata)."}\NormalTok{)}
\NormalTok{  prob\_yes}
\NormalTok{\}}

\CommentTok{\# Quick sanity test of predict}
\NormalTok{test\_pred }\OtherTok{\textless{}{-}} \FunctionTok{try}\NormalTok{(}\FunctionTok{pred\_wrapper\_prob}\NormalTok{(rf\_fit, }\FunctionTok{head}\NormalTok{(X\_train\_full, }\DecValTok{5}\NormalTok{)), }\AttributeTok{silent =} \ConstantTok{TRUE}\NormalTok{)}
\ControlFlowTok{if}\NormalTok{ (}\FunctionTok{inherits}\NormalTok{(test\_pred, }\StringTok{"try{-}error"}\NormalTok{)) }\FunctionTok{stop}\NormalTok{(}\StringTok{"Prediction wrapper test failed: "}\NormalTok{, }\FunctionTok{as.character}\NormalTok{(test\_pred))}
\FunctionTok{message}\NormalTok{(}\StringTok{"Prediction wrapper OK: returned "}\NormalTok{, }\FunctionTok{length}\NormalTok{(test\_pred), }\StringTok{" probabilities for "}\NormalTok{, }\FunctionTok{nrow}\NormalTok{(}\FunctionTok{head}\NormalTok{(X\_train\_full,}\DecValTok{5}\NormalTok{)), }\StringTok{" rows."}\NormalTok{)}
\end{Highlighting}
\end{Shaded}

\begin{verbatim}
## Prediction wrapper OK: returned 5 probabilities for 5 rows.
\end{verbatim}

\begin{Shaded}
\begin{Highlighting}[]
\CommentTok{\# SHAP computation (use data.frame for both X and newdata)}
\NormalTok{nsim }\OtherTok{\textless{}{-}} \DecValTok{20}   \CommentTok{\# 10{-}20 for quick tests; 50+ for final}

\NormalTok{explainer\_path }\OtherTok{\textless{}{-}} \StringTok{"outputs/explainer\_rf\_fastshap\_df.rds"}
\ControlFlowTok{if}\NormalTok{ (}\FunctionTok{file.exists}\NormalTok{(explainer\_path)) \{}
\NormalTok{  explainer\_rf }\OtherTok{\textless{}{-}} \FunctionTok{readRDS}\NormalTok{(explainer\_path)}
  \FunctionTok{message}\NormalTok{(}\StringTok{"Loaded saved explainer from "}\NormalTok{, explainer\_path)}
\NormalTok{\} }\ControlFlowTok{else}\NormalTok{ \{}
  \FunctionTok{message}\NormalTok{(}\StringTok{"Computing SHAP explainer with nsim = "}\NormalTok{, nsim, }\StringTok{" ..."}\NormalTok{)}
\NormalTok{  explainer\_rf }\OtherTok{\textless{}{-}}\NormalTok{ fastshap}\SpecialCharTok{::}\FunctionTok{explain}\NormalTok{(}
    \AttributeTok{object =}\NormalTok{ rf\_fit,}
    \AttributeTok{X =}\NormalTok{ X\_train\_full,            }\CommentTok{\# data.frame}
    \AttributeTok{pred\_wrapper =}\NormalTok{ pred\_wrapper\_prob,}
    \AttributeTok{nsim =}\NormalTok{ nsim,}
    \AttributeTok{adjust =} \ConstantTok{TRUE}
\NormalTok{  )}
  \FunctionTok{dir.create}\NormalTok{(}\StringTok{"outputs"}\NormalTok{, }\AttributeTok{showWarnings =} \ConstantTok{FALSE}\NormalTok{)}
  \FunctionTok{saveRDS}\NormalTok{(explainer\_rf, }\AttributeTok{file =}\NormalTok{ explainer\_path)}
  \FunctionTok{message}\NormalTok{(}\StringTok{"Saved explainer to "}\NormalTok{, explainer\_path)}
\NormalTok{\}}
\end{Highlighting}
\end{Shaded}

\begin{verbatim}
## Loaded saved explainer from outputs/explainer_rf_fastshap_df.rds
\end{verbatim}

\begin{Shaded}
\begin{Highlighting}[]
\CommentTok{\# Global importance (mean |SHAP|)}
\NormalTok{shap\_mean }\OtherTok{\textless{}{-}} \FunctionTok{as.data.frame}\NormalTok{(}\FunctionTok{abs}\NormalTok{(explainer\_rf)) }\SpecialCharTok{\%\textgreater{}\%}
  \FunctionTok{summarise\_all}\NormalTok{(mean, }\AttributeTok{na.rm =} \ConstantTok{TRUE}\NormalTok{) }\SpecialCharTok{\%\textgreater{}\%}
  \FunctionTok{pivot\_longer}\NormalTok{(}\FunctionTok{everything}\NormalTok{(), }\AttributeTok{names\_to =} \StringTok{"feature"}\NormalTok{, }\AttributeTok{values\_to =} \StringTok{"mean\_abs\_shap"}\NormalTok{) }\SpecialCharTok{\%\textgreater{}\%}
  \FunctionTok{arrange}\NormalTok{(}\FunctionTok{desc}\NormalTok{(mean\_abs\_shap))}

\FunctionTok{kable}\NormalTok{(shap\_mean, }\AttributeTok{digits =} \DecValTok{4}\NormalTok{, }\AttributeTok{caption =} \StringTok{"Global feature importance (mean |SHAP|)"}\NormalTok{) }\SpecialCharTok{\%\textgreater{}\%} \FunctionTok{kable\_styling}\NormalTok{()}
\end{Highlighting}
\end{Shaded}

\begin{longtable}[t]{lr}
\caption{\label{tab:shap-individual}Global feature importance (mean |SHAP|)}\\
\toprule
feature & mean\_abs\_shap\\
\midrule
Glucose & 0.1339\\
BMI & 0.0825\\
Age & 0.0559\\
DiabetesPedigreeFunction & 0.0406\\
Insulin & 0.0375\\
\addlinespace
Pregnancies & 0.0322\\
SkinThickness & 0.0248\\
BloodPressure & 0.0193\\
\bottomrule
\end{longtable}

\begin{Shaded}
\begin{Highlighting}[]
\CommentTok{\# Individual{-}level SHAP for a test observation (ensure obs is a data.frame)}
\NormalTok{idx }\OtherTok{\textless{}{-}} \ControlFlowTok{if}\NormalTok{ (}\FunctionTok{any}\NormalTok{(}\FunctionTok{as.character}\NormalTok{(test\_rf}\SpecialCharTok{$}\NormalTok{Outcome) }\SpecialCharTok{\%in\%} \FunctionTok{c}\NormalTok{(}\StringTok{"yes"}\NormalTok{,}\StringTok{"1"}\NormalTok{))) \{}
  \FunctionTok{which}\NormalTok{(}\FunctionTok{as.character}\NormalTok{(test\_rf}\SpecialCharTok{$}\NormalTok{Outcome) }\SpecialCharTok{==} \StringTok{"yes"}\NormalTok{)[}\DecValTok{1}\NormalTok{]}
\NormalTok{\} }\ControlFlowTok{else}\NormalTok{ \{}
  \DecValTok{1}
\NormalTok{\}}
\ControlFlowTok{if}\NormalTok{ (}\FunctionTok{is.na}\NormalTok{(idx) }\SpecialCharTok{||} \FunctionTok{length}\NormalTok{(idx) }\SpecialCharTok{==} \DecValTok{0}\NormalTok{) idx }\OtherTok{\textless{}{-}} \DecValTok{1}
\FunctionTok{message}\NormalTok{(}\StringTok{"Explaining test observation index = "}\NormalTok{, idx)}
\end{Highlighting}
\end{Shaded}

\begin{verbatim}
## Explaining test observation index = 1
\end{verbatim}

\begin{Shaded}
\begin{Highlighting}[]
\NormalTok{obs\_full }\OtherTok{\textless{}{-}}\NormalTok{ X\_test\_full[idx, , drop }\OtherTok{=} \ConstantTok{FALSE}\NormalTok{] }\SpecialCharTok{\%\textgreater{}\%} \FunctionTok{as.data.frame}\NormalTok{()  }\CommentTok{\# data.frame}

\CommentTok{\# Compute SHAP for this newdata (fastshap::explain with newdata returns SHAP for the new row)}
\NormalTok{shap\_obs\_df\_raw }\OtherTok{\textless{}{-}}\NormalTok{ fastshap}\SpecialCharTok{::}\FunctionTok{explain}\NormalTok{(}
  \AttributeTok{object =}\NormalTok{ rf\_fit,}
  \AttributeTok{X =}\NormalTok{ X\_train\_full,}
  \AttributeTok{pred\_wrapper =}\NormalTok{ pred\_wrapper\_prob,}
  \AttributeTok{nsim =}\NormalTok{ nsim,}
  \AttributeTok{newdata =}\NormalTok{ obs\_full}
\NormalTok{)}

\CommentTok{\# Convert to named vector and table}
\NormalTok{shap\_obs\_vec }\OtherTok{\textless{}{-}} \FunctionTok{as.numeric}\NormalTok{(shap\_obs\_df\_raw[}\DecValTok{1}\NormalTok{, ])}
\FunctionTok{names}\NormalTok{(shap\_obs\_vec) }\OtherTok{\textless{}{-}} \FunctionTok{colnames}\NormalTok{(shap\_obs\_df\_raw)}

\NormalTok{shap\_obs\_tbl }\OtherTok{\textless{}{-}} \FunctionTok{tibble}\NormalTok{(}\AttributeTok{feature =} \FunctionTok{names}\NormalTok{(shap\_obs\_vec), }\AttributeTok{shap =}\NormalTok{ shap\_obs\_vec) }\SpecialCharTok{\%\textgreater{}\%}
  \FunctionTok{arrange}\NormalTok{(}\FunctionTok{desc}\NormalTok{(}\FunctionTok{abs}\NormalTok{(shap))) }\SpecialCharTok{\%\textgreater{}\%}
  \FunctionTok{mutate}\NormalTok{(}\AttributeTok{direction =} \FunctionTok{ifelse}\NormalTok{(shap }\SpecialCharTok{\textgreater{}} \DecValTok{0}\NormalTok{, }\StringTok{"increases risk"}\NormalTok{, }\StringTok{"decreases risk"}\NormalTok{))}

\FunctionTok{kable}\NormalTok{(shap\_obs\_tbl }\SpecialCharTok{\%\textgreater{}\%} \FunctionTok{head}\NormalTok{(}\DecValTok{10}\NormalTok{), }\AttributeTok{digits =} \DecValTok{4}\NormalTok{,}
      \AttributeTok{caption =} \FunctionTok{paste0}\NormalTok{(}\StringTok{"Top SHAP contributors for test index = "}\NormalTok{, idx)) }\SpecialCharTok{\%\textgreater{}\%}
  \FunctionTok{kable\_styling}\NormalTok{(}\AttributeTok{full\_width =} \ConstantTok{FALSE}\NormalTok{)}
\end{Highlighting}
\end{Shaded}

\begin{longtable}[t]{lrl}
\caption{\label{tab:shap-individual}Top SHAP contributors for test index = 1}\\
\toprule
feature & shap & direction\\
\midrule
Glucose & 0.1997 & increases risk\\
BMI & 0.0741 & increases risk\\
Insulin & 0.0417 & increases risk\\
DiabetesPedigreeFunction & 0.0409 & increases risk\\
Pregnancies & -0.0375 & decreases risk\\
\addlinespace
SkinThickness & 0.0318 & increases risk\\
Age & 0.0149 & increases risk\\
BloodPressure & 0.0100 & increases risk\\
\bottomrule
\end{longtable}

\begin{Shaded}
\begin{Highlighting}[]
\FunctionTok{cat}\NormalTok{(}\StringTok{"}\SpecialCharTok{\textbackslash{}n}\StringTok{Observation predictor values:}\SpecialCharTok{\textbackslash{}n}\StringTok{"}\NormalTok{)}
\end{Highlighting}
\end{Shaded}

\begin{verbatim}
## 
## Observation predictor values:
\end{verbatim}

\begin{Shaded}
\begin{Highlighting}[]
\FunctionTok{print}\NormalTok{(obs\_full)}
\end{Highlighting}
\end{Shaded}

\begin{verbatim}
##   Pregnancies Glucose BloodPressure SkinThickness Insulin  BMI
## 1           6     148            72            35     125 33.6
##   DiabetesPedigreeFunction Age
## 1                    0.627  50
\end{verbatim}

\begin{Shaded}
\begin{Highlighting}[]
\CommentTok{\# Plot contributions}
\NormalTok{shap\_obs\_tbl }\SpecialCharTok{\%\textgreater{}\%}
  \FunctionTok{head}\NormalTok{(}\DecValTok{10}\NormalTok{) }\SpecialCharTok{\%\textgreater{}\%}
  \FunctionTok{mutate}\NormalTok{(}\AttributeTok{feature =} \FunctionTok{factor}\NormalTok{(feature, }\AttributeTok{levels =} \FunctionTok{rev}\NormalTok{(feature))) }\SpecialCharTok{\%\textgreater{}\%}
  \FunctionTok{ggplot}\NormalTok{(}\FunctionTok{aes}\NormalTok{(}\AttributeTok{x =}\NormalTok{ shap, }\AttributeTok{y =}\NormalTok{ feature, }\AttributeTok{fill =}\NormalTok{ shap }\SpecialCharTok{\textgreater{}} \DecValTok{0}\NormalTok{)) }\SpecialCharTok{+}
  \FunctionTok{geom\_col}\NormalTok{() }\SpecialCharTok{+}
  \FunctionTok{scale\_fill\_manual}\NormalTok{(}\AttributeTok{values =} \FunctionTok{c}\NormalTok{(}\StringTok{"TRUE"} \OtherTok{=} \StringTok{"salmon"}\NormalTok{, }\StringTok{"FALSE"} \OtherTok{=} \StringTok{"skyblue"}\NormalTok{), }\AttributeTok{guide =} \ConstantTok{FALSE}\NormalTok{) }\SpecialCharTok{+}
  \FunctionTok{labs}\NormalTok{(}\AttributeTok{title =} \FunctionTok{paste0}\NormalTok{(}\StringTok{"SHAP contributions for test index "}\NormalTok{, idx),}
       \AttributeTok{x =} \StringTok{"SHAP value (positive {-}\textgreater{} increases predicted probability)"}\NormalTok{, }\AttributeTok{y =} \ConstantTok{NULL}\NormalTok{) }\SpecialCharTok{+}
  \FunctionTok{theme\_minimal}\NormalTok{(}\AttributeTok{base\_size =} \DecValTok{12}\NormalTok{)}
\end{Highlighting}
\end{Shaded}

\begin{verbatim}
## Warning: The `guide` argument in `scale_*()` cannot be `FALSE`. This was deprecated in
## ggplot2 3.3.4.
## i Please use "none" instead.
## This warning is displayed once every 8 hours.
## Call `lifecycle::last_lifecycle_warnings()` to see where this warning was
## generated.
\end{verbatim}

\includegraphics{Final-Projet_files/figure-latex/shap-individual-1.pdf}

Expected output description: - Table with top features pushing the
prediction up (positive shap) or down (negative shap) for the selected
individual. Useful to show during presentation.

\subsection{Statistical tests \& model
significance}\label{statistical-tests-model-significance}

Description: We perform basic statistical tests such as Wald tests for
important logistic regression coefficients and (optionally) DeLong test
to compare ROC AUCs.

Expected output: - p-values for logistic coefficients and a short
comment on which variables are statistically significant.

\begin{Shaded}
\begin{Highlighting}[]
\CommentTok{\# Wald p{-}values from summary(glm\_fit)}
\NormalTok{coefs }\OtherTok{\textless{}{-}} \FunctionTok{summary}\NormalTok{(glm\_fit)}\SpecialCharTok{$}\NormalTok{coefficients}
\NormalTok{coefs }\SpecialCharTok{\%\textgreater{}\%} \FunctionTok{as.data.frame}\NormalTok{() }\SpecialCharTok{\%\textgreater{}\%} \FunctionTok{rownames\_to\_column}\NormalTok{(}\StringTok{"term"}\NormalTok{) }\SpecialCharTok{\%\textgreater{}\%}
  \FunctionTok{kable}\NormalTok{(}\AttributeTok{caption =} \StringTok{"Logistic regression coefficients (Wald test)"}\NormalTok{) }\SpecialCharTok{\%\textgreater{}\%}
  \FunctionTok{kable\_styling}\NormalTok{(}\AttributeTok{full\_width =} \ConstantTok{FALSE}\NormalTok{)}
\end{Highlighting}
\end{Shaded}

\begin{longtable}[t]{lrrrr}
\caption{\label{tab:stats-tests}Logistic regression coefficients (Wald test)}\\
\toprule
term & Estimate & Std. Error & z value & Pr(>\&\#124;z\&\#124;)\\
\midrule
(Intercept) & -9.4197232 & 0.9510946 & -9.9040867 & 0.0000000\\
Pregnancies & 0.1168057 & 0.0379992 & 3.0738990 & 0.0021128\\
Glucose & 0.0404826 & 0.0047280 & 8.5623300 & 0.0000000\\
BloodPressure & -0.0087293 & 0.0100446 & -0.8690516 & 0.3848189\\
SkinThickness & 0.0094087 & 0.0153902 & 0.6113429 & 0.5409726\\
\addlinespace
Insulin & -0.0012471 & 0.0012732 & -0.9794744 & 0.3273456\\
BMI & 0.0909910 & 0.0209989 & 4.3331292 & 0.0000147\\
DiabetesPedigreeFunction & 0.6618850 & 0.3312111 & 1.9983781 & 0.0456757\\
Age & 0.0121602 & 0.0111643 & 1.0892058 & 0.2760631\\
\bottomrule
\end{longtable}

\begin{Shaded}
\begin{Highlighting}[]
\CommentTok{\# Compare AUCs (DeLong) between RF and XGBoost}
\CommentTok{\# pROC::roc.test(roc\_rf, roc\_xgb)  \# uncomment to run; may give p{-}value whether difference significant}
\end{Highlighting}
\end{Shaded}

Further Exploration

\begin{center}\rule{0.5\linewidth}{0.5pt}\end{center}

Description: This supplemental RMarkdown module performs (A) a full
multiple-imputation workflow using mice, compares predictive performance
across imputations to the simple median-imputed baseline, and (B) adds
several descriptive EDA visualizations that the rubric explicitly
favored (AgeGroup counts, outcome-stratified plots, BMI-category
breakdowns). The code is written so you can either knit this file
standalone or insert the chunks into your main \texttt{proposal.Rmd} /
final report.

Where to add into the final document (be specific): - Insert the entire
``Multiple imputation (mice) analysis'' section immediately after the
``Simple imputation (median) --- reproducible baseline'' chunk in your
main document (i.e., after the chunk labeled \texttt{simple-impute}). -
Insert the ``Extended EDA visualizations'' chunk(s) into the main EDA
section (after the current \texttt{eda-plots} chunk). If you prefer, you
can paste these chunks into the EDA section in place of or in addition
to the current plots. - Remove or replace the earlier commented
\texttt{mice} chunk with this working multiple-imputation section. - Use
the chunk labels provided here (they are unique) when you paste into the
main Rmd so knit order and caching are consistent.

How this addresses the rubric/deductions: - Runs \texttt{mice} (multiple
imputation) and shows comparison to simple median imputation,
eliminating the ``commented out'' issue. - Adds AgeGroup counts and
additional outcome-stratified visualizations to improve descriptive EDA
and address the minor polish deductions.

Notes about runtime: - \texttt{mice} with default settings and m = 5 is
moderate in runtime (minutes). If you need faster runs for demos, reduce
\texttt{m} to 3 or set \texttt{maxit\ =\ 10}. - SHAP and large
hyperparameter grids are more expensive; these chunks are modest and
focused on imputation/EDA.

\subsection{Code and narrative}\label{code-and-narrative}

\begin{Shaded}
\begin{Highlighting}[]
\CommentTok{\# This chunk prepares data if the main document hasn\textquotesingle{}t already prepared \textquotesingle{}diabetes\_clean\textquotesingle{}.}
\CommentTok{\# If diabetes\_clean exists from your main Rmd, this will use it; otherwise it reads the CSV and}
\CommentTok{\# applies the zero{-}\textgreater{}NA step and derived features so this module is self{-}contained.}

\ControlFlowTok{if}\NormalTok{(}\SpecialCharTok{!}\FunctionTok{exists}\NormalTok{(}\StringTok{"diabetes\_clean"}\NormalTok{)) \{}
\NormalTok{  csv\_path }\OtherTok{\textless{}{-}} \StringTok{"diabetes.csv"}
  \ControlFlowTok{if}\NormalTok{(}\SpecialCharTok{!}\FunctionTok{file.exists}\NormalTok{(csv\_path)) }\FunctionTok{stop}\NormalTok{(}\StringTok{"Please place \textasciigrave{}diabetes.csv\textasciigrave{} in the working directory before running this module."}\NormalTok{)}
\NormalTok{  diabetes }\OtherTok{\textless{}{-}}\NormalTok{ readr}\SpecialCharTok{::}\FunctionTok{read\_csv}\NormalTok{(csv\_path, }\AttributeTok{show\_col\_types =} \ConstantTok{FALSE}\NormalTok{)}

\NormalTok{  zero\_as\_na\_cols }\OtherTok{\textless{}{-}} \FunctionTok{c}\NormalTok{(}\StringTok{"Glucose"}\NormalTok{,}\StringTok{"BloodPressure"}\NormalTok{,}\StringTok{"SkinThickness"}\NormalTok{,}\StringTok{"Insulin"}\NormalTok{,}\StringTok{"BMI"}\NormalTok{)}
\NormalTok{  diabetes\_clean }\OtherTok{\textless{}{-}}\NormalTok{ diabetes }\SpecialCharTok{\%\textgreater{}\%}
    \FunctionTok{mutate}\NormalTok{(}\FunctionTok{across}\NormalTok{(}\FunctionTok{all\_of}\NormalTok{(zero\_as\_na\_cols), }\SpecialCharTok{\textasciitilde{}} \FunctionTok{na\_if}\NormalTok{(., }\DecValTok{0}\NormalTok{))) }\SpecialCharTok{\%\textgreater{}\%}
    \FunctionTok{mutate}\NormalTok{(}
      \AttributeTok{AgeGroup =} \FunctionTok{case\_when}\NormalTok{(}
\NormalTok{        Age }\SpecialCharTok{\textless{}=} \DecValTok{30} \SpecialCharTok{\textasciitilde{}} \StringTok{"21{-}30"}\NormalTok{,}
\NormalTok{        Age }\SpecialCharTok{\textless{}=} \DecValTok{40} \SpecialCharTok{\textasciitilde{}} \StringTok{"31{-}40"}\NormalTok{,}
\NormalTok{        Age }\SpecialCharTok{\textless{}=} \DecValTok{50} \SpecialCharTok{\textasciitilde{}} \StringTok{"41{-}50"}\NormalTok{,}
        \ConstantTok{TRUE}      \SpecialCharTok{\textasciitilde{}} \StringTok{"51+"}
\NormalTok{      ),}
      \AttributeTok{Insulin\_log =} \FunctionTok{log1p}\NormalTok{(Insulin),}
      \AttributeTok{BMI\_cat =} \FunctionTok{cut}\NormalTok{(BMI, }\AttributeTok{breaks =} \FunctionTok{c}\NormalTok{(}\DecValTok{0}\NormalTok{,}\FloatTok{18.5}\NormalTok{,}\DecValTok{25}\NormalTok{,}\DecValTok{30}\NormalTok{,}\ConstantTok{Inf}\NormalTok{), }\AttributeTok{labels =} \FunctionTok{c}\NormalTok{(}\StringTok{"Underweight"}\NormalTok{,}\StringTok{"Normal"}\NormalTok{,}\StringTok{"Overweight"}\NormalTok{,}\StringTok{"Obese"}\NormalTok{))}
\NormalTok{    )}
\NormalTok{\}}

\CommentTok{\# Show a small verification table}
\NormalTok{knitr}\SpecialCharTok{::}\FunctionTok{kable}\NormalTok{(}\FunctionTok{head}\NormalTok{(diabetes\_clean), }\AttributeTok{caption =} \StringTok{"Preview of diabetes\_clean (used for mice)"}\NormalTok{)}
\end{Highlighting}
\end{Shaded}

\begin{longtable}[]{@{}
  >{\raggedleft\arraybackslash}p{(\columnwidth - 22\tabcolsep) * \real{0.0923}}
  >{\raggedleft\arraybackslash}p{(\columnwidth - 22\tabcolsep) * \real{0.0615}}
  >{\raggedleft\arraybackslash}p{(\columnwidth - 22\tabcolsep) * \real{0.1077}}
  >{\raggedleft\arraybackslash}p{(\columnwidth - 22\tabcolsep) * \real{0.1077}}
  >{\raggedleft\arraybackslash}p{(\columnwidth - 22\tabcolsep) * \real{0.0615}}
  >{\raggedleft\arraybackslash}p{(\columnwidth - 22\tabcolsep) * \real{0.0385}}
  >{\raggedleft\arraybackslash}p{(\columnwidth - 22\tabcolsep) * \real{0.1923}}
  >{\raggedleft\arraybackslash}p{(\columnwidth - 22\tabcolsep) * \real{0.0308}}
  >{\raggedleft\arraybackslash}p{(\columnwidth - 22\tabcolsep) * \real{0.0615}}
  >{\raggedright\arraybackslash}p{(\columnwidth - 22\tabcolsep) * \real{0.0692}}
  >{\raggedleft\arraybackslash}p{(\columnwidth - 22\tabcolsep) * \real{0.0923}}
  >{\raggedright\arraybackslash}p{(\columnwidth - 22\tabcolsep) * \real{0.0846}}@{}}
\caption{Preview of diabetes\_clean (used for mice)}\tabularnewline
\toprule\noalign{}
\begin{minipage}[b]{\linewidth}\raggedleft
Pregnancies
\end{minipage} & \begin{minipage}[b]{\linewidth}\raggedleft
Glucose
\end{minipage} & \begin{minipage}[b]{\linewidth}\raggedleft
BloodPressure
\end{minipage} & \begin{minipage}[b]{\linewidth}\raggedleft
SkinThickness
\end{minipage} & \begin{minipage}[b]{\linewidth}\raggedleft
Insulin
\end{minipage} & \begin{minipage}[b]{\linewidth}\raggedleft
BMI
\end{minipage} & \begin{minipage}[b]{\linewidth}\raggedleft
DiabetesPedigreeFunction
\end{minipage} & \begin{minipage}[b]{\linewidth}\raggedleft
Age
\end{minipage} & \begin{minipage}[b]{\linewidth}\raggedleft
Outcome
\end{minipage} & \begin{minipage}[b]{\linewidth}\raggedright
AgeGroup
\end{minipage} & \begin{minipage}[b]{\linewidth}\raggedleft
Insulin\_log
\end{minipage} & \begin{minipage}[b]{\linewidth}\raggedright
BMI\_cat
\end{minipage} \\
\midrule\noalign{}
\endfirsthead
\toprule\noalign{}
\begin{minipage}[b]{\linewidth}\raggedleft
Pregnancies
\end{minipage} & \begin{minipage}[b]{\linewidth}\raggedleft
Glucose
\end{minipage} & \begin{minipage}[b]{\linewidth}\raggedleft
BloodPressure
\end{minipage} & \begin{minipage}[b]{\linewidth}\raggedleft
SkinThickness
\end{minipage} & \begin{minipage}[b]{\linewidth}\raggedleft
Insulin
\end{minipage} & \begin{minipage}[b]{\linewidth}\raggedleft
BMI
\end{minipage} & \begin{minipage}[b]{\linewidth}\raggedleft
DiabetesPedigreeFunction
\end{minipage} & \begin{minipage}[b]{\linewidth}\raggedleft
Age
\end{minipage} & \begin{minipage}[b]{\linewidth}\raggedleft
Outcome
\end{minipage} & \begin{minipage}[b]{\linewidth}\raggedright
AgeGroup
\end{minipage} & \begin{minipage}[b]{\linewidth}\raggedleft
Insulin\_log
\end{minipage} & \begin{minipage}[b]{\linewidth}\raggedright
BMI\_cat
\end{minipage} \\
\midrule\noalign{}
\endhead
\bottomrule\noalign{}
\endlastfoot
6 & 148 & 72 & 35 & NA & 33.6 & 0.627 & 50 & 1 & 41-50 & NA & Obese \\
1 & 85 & 66 & 29 & NA & 26.6 & 0.351 & 31 & 0 & 31-40 & NA &
Overweight \\
8 & 183 & 64 & NA & NA & 23.3 & 0.672 & 32 & 1 & 31-40 & NA & Normal \\
1 & 89 & 66 & 23 & 94 & 28.1 & 0.167 & 21 & 0 & 21-30 & 4.553877 &
Overweight \\
0 & 137 & 40 & 35 & 168 & 43.1 & 2.288 & 33 & 1 & 31-40 & 5.129899 &
Obese \\
5 & 116 & 74 & NA & NA & 25.6 & 0.201 & 30 & 0 & 21-30 & NA &
Overweight \\
\end{longtable}

\subsection{Multiple imputation with mice and pooled logistic
regression}\label{multiple-imputation-with-mice-and-pooled-logistic-regression}

Description: - Use \texttt{mice} (m = 5 imputations by default) to
impute missing values for numeric predictors where zeros were replaced
by \texttt{NA}. - Fit logistic regression across imputations and pool
coefficient estimates (Wald statistics, ORs). - Compare pooled logistic
coefficients with the median-impute logistic results (if median-impute
model exists, this chunk will compute baseline here to be safe).

Expected outputs: - \texttt{mice} diagnostics (convergence plot
optional). - Pooled logistic regression coefficient table with OR and
95\% CI. - Mean and SD of test AUCs computed across completed imputed
datasets (gives predictive-performance sense for imputation
variability). - A small commentary block summarizing differences vs
median-imputation.

\begin{Shaded}
\begin{Highlighting}[]
\FunctionTok{library}\NormalTok{(mice)}
\FunctionTok{library}\NormalTok{(dplyr)}
\FunctionTok{library}\NormalTok{(broom)}
\FunctionTok{library}\NormalTok{(kableExtra)}

\CommentTok{\# Variables to impute (include Outcome)}
\NormalTok{imp\_vars }\OtherTok{\textless{}{-}} \FunctionTok{c}\NormalTok{(}\StringTok{"Pregnancies"}\NormalTok{,}\StringTok{"Glucose"}\NormalTok{,}\StringTok{"BloodPressure"}\NormalTok{,}\StringTok{"SkinThickness"}\NormalTok{,}\StringTok{"Insulin"}\NormalTok{,}\StringTok{"BMI"}\NormalTok{,}
              \StringTok{"DiabetesPedigreeFunction"}\NormalTok{,}\StringTok{"Age"}\NormalTok{,}\StringTok{"Outcome"}\NormalTok{)}

\CommentTok{\# Prepare data for mice; convert Outcome to numeric 0/1 if it\textquotesingle{}s factor}
\NormalTok{di\_for\_mice }\OtherTok{\textless{}{-}}\NormalTok{ diabetes\_clean }\SpecialCharTok{\%\textgreater{}\%}
  \FunctionTok{select}\NormalTok{(}\FunctionTok{all\_of}\NormalTok{(imp\_vars)) }\SpecialCharTok{\%\textgreater{}\%}
  \FunctionTok{mutate}\NormalTok{(}\AttributeTok{Outcome =} \FunctionTok{as.numeric}\NormalTok{(}\FunctionTok{as.character}\NormalTok{(Outcome)))}

\CommentTok{\# Run mice: 5 imputations, predictive mean matching}
\FunctionTok{set.seed}\NormalTok{(}\DecValTok{123}\NormalTok{)}
\NormalTok{mi }\OtherTok{\textless{}{-}} \FunctionTok{mice}\NormalTok{(di\_for\_mice, }\AttributeTok{m =} \DecValTok{5}\NormalTok{, }\AttributeTok{method =} \StringTok{"pmm"}\NormalTok{, }\AttributeTok{maxit =} \DecValTok{20}\NormalTok{, }\AttributeTok{seed =} \DecValTok{123}\NormalTok{, }\AttributeTok{printFlag =} \ConstantTok{TRUE}\NormalTok{)}
\end{Highlighting}
\end{Shaded}

\begin{verbatim}
## 
##  iter imp variable
##   1   1  Glucose  BloodPressure  SkinThickness  Insulin  BMI
##   1   2  Glucose  BloodPressure  SkinThickness  Insulin  BMI
##   1   3  Glucose  BloodPressure  SkinThickness  Insulin  BMI
##   1   4  Glucose  BloodPressure  SkinThickness  Insulin  BMI
##   1   5  Glucose  BloodPressure  SkinThickness  Insulin  BMI
##   2   1  Glucose  BloodPressure  SkinThickness  Insulin  BMI
##   2   2  Glucose  BloodPressure  SkinThickness  Insulin  BMI
##   2   3  Glucose  BloodPressure  SkinThickness  Insulin  BMI
##   2   4  Glucose  BloodPressure  SkinThickness  Insulin  BMI
##   2   5  Glucose  BloodPressure  SkinThickness  Insulin  BMI
##   3   1  Glucose  BloodPressure  SkinThickness  Insulin  BMI
##   3   2  Glucose  BloodPressure  SkinThickness  Insulin  BMI
##   3   3  Glucose  BloodPressure  SkinThickness  Insulin  BMI
##   3   4  Glucose  BloodPressure  SkinThickness  Insulin  BMI
##   3   5  Glucose  BloodPressure  SkinThickness  Insulin  BMI
##   4   1  Glucose  BloodPressure  SkinThickness  Insulin  BMI
##   4   2  Glucose  BloodPressure  SkinThickness  Insulin  BMI
##   4   3  Glucose  BloodPressure  SkinThickness  Insulin  BMI
##   4   4  Glucose  BloodPressure  SkinThickness  Insulin  BMI
##   4   5  Glucose  BloodPressure  SkinThickness  Insulin  BMI
##   5   1  Glucose  BloodPressure  SkinThickness  Insulin  BMI
##   5   2  Glucose  BloodPressure  SkinThickness  Insulin  BMI
##   5   3  Glucose  BloodPressure  SkinThickness  Insulin  BMI
##   5   4  Glucose  BloodPressure  SkinThickness  Insulin  BMI
##   5   5  Glucose  BloodPressure  SkinThickness  Insulin  BMI
##   6   1  Glucose  BloodPressure  SkinThickness  Insulin  BMI
##   6   2  Glucose  BloodPressure  SkinThickness  Insulin  BMI
##   6   3  Glucose  BloodPressure  SkinThickness  Insulin  BMI
##   6   4  Glucose  BloodPressure  SkinThickness  Insulin  BMI
##   6   5  Glucose  BloodPressure  SkinThickness  Insulin  BMI
##   7   1  Glucose  BloodPressure  SkinThickness  Insulin  BMI
##   7   2  Glucose  BloodPressure  SkinThickness  Insulin  BMI
##   7   3  Glucose  BloodPressure  SkinThickness  Insulin  BMI
##   7   4  Glucose  BloodPressure  SkinThickness  Insulin  BMI
##   7   5  Glucose  BloodPressure  SkinThickness  Insulin  BMI
##   8   1  Glucose  BloodPressure  SkinThickness  Insulin  BMI
##   8   2  Glucose  BloodPressure  SkinThickness  Insulin  BMI
##   8   3  Glucose  BloodPressure  SkinThickness  Insulin  BMI
##   8   4  Glucose  BloodPressure  SkinThickness  Insulin  BMI
##   8   5  Glucose  BloodPressure  SkinThickness  Insulin  BMI
##   9   1  Glucose  BloodPressure  SkinThickness  Insulin  BMI
##   9   2  Glucose  BloodPressure  SkinThickness  Insulin  BMI
##   9   3  Glucose  BloodPressure  SkinThickness  Insulin  BMI
##   9   4  Glucose  BloodPressure  SkinThickness  Insulin  BMI
##   9   5  Glucose  BloodPressure  SkinThickness  Insulin  BMI
##   10   1  Glucose  BloodPressure  SkinThickness  Insulin  BMI
##   10   2  Glucose  BloodPressure  SkinThickness  Insulin  BMI
##   10   3  Glucose  BloodPressure  SkinThickness  Insulin  BMI
##   10   4  Glucose  BloodPressure  SkinThickness  Insulin  BMI
##   10   5  Glucose  BloodPressure  SkinThickness  Insulin  BMI
##   11   1  Glucose  BloodPressure  SkinThickness  Insulin  BMI
##   11   2  Glucose  BloodPressure  SkinThickness  Insulin  BMI
##   11   3  Glucose  BloodPressure  SkinThickness  Insulin  BMI
##   11   4  Glucose  BloodPressure  SkinThickness  Insulin  BMI
##   11   5  Glucose  BloodPressure  SkinThickness  Insulin  BMI
##   12   1  Glucose  BloodPressure  SkinThickness  Insulin  BMI
##   12   2  Glucose  BloodPressure  SkinThickness  Insulin  BMI
##   12   3  Glucose  BloodPressure  SkinThickness  Insulin  BMI
##   12   4  Glucose  BloodPressure  SkinThickness  Insulin  BMI
##   12   5  Glucose  BloodPressure  SkinThickness  Insulin  BMI
##   13   1  Glucose  BloodPressure  SkinThickness  Insulin  BMI
##   13   2  Glucose  BloodPressure  SkinThickness  Insulin  BMI
##   13   3  Glucose  BloodPressure  SkinThickness  Insulin  BMI
##   13   4  Glucose  BloodPressure  SkinThickness  Insulin  BMI
##   13   5  Glucose  BloodPressure  SkinThickness  Insulin  BMI
##   14   1  Glucose  BloodPressure  SkinThickness  Insulin  BMI
##   14   2  Glucose  BloodPressure  SkinThickness  Insulin  BMI
##   14   3  Glucose  BloodPressure  SkinThickness  Insulin  BMI
##   14   4  Glucose  BloodPressure  SkinThickness  Insulin  BMI
##   14   5  Glucose  BloodPressure  SkinThickness  Insulin  BMI
##   15   1  Glucose  BloodPressure  SkinThickness  Insulin  BMI
##   15   2  Glucose  BloodPressure  SkinThickness  Insulin  BMI
##   15   3  Glucose  BloodPressure  SkinThickness  Insulin  BMI
##   15   4  Glucose  BloodPressure  SkinThickness  Insulin  BMI
##   15   5  Glucose  BloodPressure  SkinThickness  Insulin  BMI
##   16   1  Glucose  BloodPressure  SkinThickness  Insulin  BMI
##   16   2  Glucose  BloodPressure  SkinThickness  Insulin  BMI
##   16   3  Glucose  BloodPressure  SkinThickness  Insulin  BMI
##   16   4  Glucose  BloodPressure  SkinThickness  Insulin  BMI
##   16   5  Glucose  BloodPressure  SkinThickness  Insulin  BMI
##   17   1  Glucose  BloodPressure  SkinThickness  Insulin  BMI
##   17   2  Glucose  BloodPressure  SkinThickness  Insulin  BMI
##   17   3  Glucose  BloodPressure  SkinThickness  Insulin  BMI
##   17   4  Glucose  BloodPressure  SkinThickness  Insulin  BMI
##   17   5  Glucose  BloodPressure  SkinThickness  Insulin  BMI
##   18   1  Glucose  BloodPressure  SkinThickness  Insulin  BMI
##   18   2  Glucose  BloodPressure  SkinThickness  Insulin  BMI
##   18   3  Glucose  BloodPressure  SkinThickness  Insulin  BMI
##   18   4  Glucose  BloodPressure  SkinThickness  Insulin  BMI
##   18   5  Glucose  BloodPressure  SkinThickness  Insulin  BMI
##   19   1  Glucose  BloodPressure  SkinThickness  Insulin  BMI
##   19   2  Glucose  BloodPressure  SkinThickness  Insulin  BMI
##   19   3  Glucose  BloodPressure  SkinThickness  Insulin  BMI
##   19   4  Glucose  BloodPressure  SkinThickness  Insulin  BMI
##   19   5  Glucose  BloodPressure  SkinThickness  Insulin  BMI
##   20   1  Glucose  BloodPressure  SkinThickness  Insulin  BMI
##   20   2  Glucose  BloodPressure  SkinThickness  Insulin  BMI
##   20   3  Glucose  BloodPressure  SkinThickness  Insulin  BMI
##   20   4  Glucose  BloodPressure  SkinThickness  Insulin  BMI
##   20   5  Glucose  BloodPressure  SkinThickness  Insulin  BMI
\end{verbatim}

\begin{Shaded}
\begin{Highlighting}[]
\CommentTok{\# Optional traceplot for diagnostics}
\CommentTok{\# plot(mi, c("Glucose", "Insulin"))}

\CommentTok{\# Fit logistic regression on each imputed dataset}
\NormalTok{fit\_mi }\OtherTok{\textless{}{-}} \FunctionTok{with}\NormalTok{(mi, }\FunctionTok{glm}\NormalTok{(Outcome }\SpecialCharTok{\textasciitilde{}}\NormalTok{ Pregnancies }\SpecialCharTok{+}\NormalTok{ Glucose }\SpecialCharTok{+}\NormalTok{ BloodPressure }\SpecialCharTok{+}\NormalTok{ SkinThickness }\SpecialCharTok{+}
\NormalTok{                        Insulin }\SpecialCharTok{+}\NormalTok{ BMI }\SpecialCharTok{+}\NormalTok{ DiabetesPedigreeFunction }\SpecialCharTok{+}\NormalTok{ Age, }\AttributeTok{family =}\NormalTok{ binomial))}

\CommentTok{\# Pool results}
\NormalTok{pooled }\OtherTok{\textless{}{-}} \FunctionTok{pool}\NormalTok{(fit\_mi)}

\CommentTok{\# Summarize results with ORs}
\NormalTok{summary\_pooled }\OtherTok{\textless{}{-}} \FunctionTok{summary}\NormalTok{(pooled, }\AttributeTok{conf.int =} \ConstantTok{TRUE}\NormalTok{, }\AttributeTok{exponentiate =} \ConstantTok{TRUE}\NormalTok{)}

\CommentTok{\# Clean up table for display}
\NormalTok{summary\_pooled }\SpecialCharTok{\%\textgreater{}\%}
\NormalTok{  dplyr}\SpecialCharTok{::}\FunctionTok{select}\NormalTok{(term, estimate, std.error, }\StringTok{\textasciigrave{}}\AttributeTok{2.5 \%}\StringTok{\textasciigrave{}}\NormalTok{, }\StringTok{\textasciigrave{}}\AttributeTok{97.5 \%}\StringTok{\textasciigrave{}}\NormalTok{, p.value) }\SpecialCharTok{\%\textgreater{}\%}
\NormalTok{  dplyr}\SpecialCharTok{::}\FunctionTok{rename}\NormalTok{(}\AttributeTok{OR =}\NormalTok{ estimate, }\AttributeTok{OR\_low =} \StringTok{\textasciigrave{}}\AttributeTok{2.5 \%}\StringTok{\textasciigrave{}}\NormalTok{, }\AttributeTok{OR\_high =} \StringTok{\textasciigrave{}}\AttributeTok{97.5 \%}\StringTok{\textasciigrave{}}\NormalTok{) }\SpecialCharTok{\%\textgreater{}\%}
  \FunctionTok{kable}\NormalTok{(}\AttributeTok{digits =} \DecValTok{3}\NormalTok{, }\AttributeTok{caption =} \StringTok{"Pooled logistic regression results (mice, ORs)"}\NormalTok{) }\SpecialCharTok{\%\textgreater{}\%}
  \FunctionTok{kable\_styling}\NormalTok{(}\AttributeTok{full\_width =} \ConstantTok{FALSE}\NormalTok{)}
\end{Highlighting}
\end{Shaded}

\begin{longtable}[t]{lrrrrr}
\caption{\label{tab:mice-impute-run}Pooled logistic regression results (mice, ORs)}\\
\toprule
term & OR & std.error & OR\_low & OR\_high & p.value\\
\midrule
(Intercept) & 0.000 & 0.892 & 0.000 & 0.001 & 0.000\\
Pregnancies & 1.127 & 0.034 & 1.055 & 1.204 & 0.000\\
Glucose & 1.040 & 0.005 & 1.029 & 1.050 & 0.000\\
BloodPressure & 0.989 & 0.008 & 0.973 & 1.006 & 0.213\\
SkinThickness & 1.006 & 0.013 & 0.981 & 1.031 & 0.660\\
\addlinespace
Insulin & 0.999 & 0.002 & 0.995 & 1.002 & 0.515\\
BMI & 1.097 & 0.020 & 1.054 & 1.141 & 0.000\\
DiabetesPedigreeFunction & 2.372 & 0.298 & 1.322 & 4.257 & 0.004\\
Age & 1.015 & 0.010 & 0.995 & 1.035 & 0.142\\
\bottomrule
\end{longtable}

\subsection{Compare predictive performance across imputed datasets
(AUC)}\label{compare-predictive-performance-across-imputed-datasets-auc}

Description: - For each completed imputed dataset, we will: - Recreate a
consistent train/test split (same seed so splits are consistent across
imputations). - Fit a logistic regression on the training portion and
compute AUC on the test set. - Report mean and SD of AUC across the m
imputations to show uncertainty introduced by imputation.

Expected outputs: - A small table of per-imputation AUCs and a mean ± SD
line. This demonstrates whether imputation materially changes predictive
performance.

\begin{Shaded}
\begin{Highlighting}[]
\FunctionTok{library}\NormalTok{(pROC)}
\FunctionTok{library}\NormalTok{(caret)}

\FunctionTok{set.seed}\NormalTok{(}\DecValTok{123}\NormalTok{)}

\CommentTok{\# FIX: explicitly define m}
\NormalTok{m }\OtherTok{\textless{}{-}}\NormalTok{ mi}\SpecialCharTok{$}\NormalTok{m}

\NormalTok{train\_index\_mice }\OtherTok{\textless{}{-}}\NormalTok{ caret}\SpecialCharTok{::}\FunctionTok{createDataPartition}\NormalTok{(}
\NormalTok{  di\_for\_mice}\SpecialCharTok{$}\NormalTok{Outcome, }\AttributeTok{p =} \FloatTok{0.75}\NormalTok{, }\AttributeTok{list =} \ConstantTok{FALSE}
\NormalTok{)}

\NormalTok{auc\_vals }\OtherTok{\textless{}{-}} \FunctionTok{numeric}\NormalTok{(m)}

\ControlFlowTok{for}\NormalTok{(i }\ControlFlowTok{in} \DecValTok{1}\SpecialCharTok{:}\NormalTok{m) \{}
\NormalTok{  comp }\OtherTok{\textless{}{-}} \FunctionTok{complete}\NormalTok{(mi, i)}

\NormalTok{  train\_mi }\OtherTok{\textless{}{-}}\NormalTok{ comp[train\_index\_mice, ]}
\NormalTok{  test\_mi  }\OtherTok{\textless{}{-}}\NormalTok{ comp[}\SpecialCharTok{{-}}\NormalTok{train\_index\_mice, ]}

\NormalTok{  glm\_i }\OtherTok{\textless{}{-}} \FunctionTok{glm}\NormalTok{(}
\NormalTok{    Outcome }\SpecialCharTok{\textasciitilde{}}\NormalTok{ Pregnancies }\SpecialCharTok{+}\NormalTok{ Glucose }\SpecialCharTok{+}\NormalTok{ BloodPressure }\SpecialCharTok{+}\NormalTok{ SkinThickness }\SpecialCharTok{+}
\NormalTok{      Insulin }\SpecialCharTok{+}\NormalTok{ BMI }\SpecialCharTok{+}\NormalTok{ DiabetesPedigreeFunction }\SpecialCharTok{+}\NormalTok{ Age,}
    \AttributeTok{data =}\NormalTok{ train\_mi,}
    \AttributeTok{family =}\NormalTok{ binomial}
\NormalTok{  )}

\NormalTok{  preds }\OtherTok{\textless{}{-}} \FunctionTok{predict}\NormalTok{(glm\_i, }\AttributeTok{newdata =}\NormalTok{ test\_mi, }\AttributeTok{type =} \StringTok{"response"}\NormalTok{)}
\NormalTok{  roc\_i }\OtherTok{\textless{}{-}}\NormalTok{ pROC}\SpecialCharTok{::}\FunctionTok{roc}\NormalTok{(test\_mi}\SpecialCharTok{$}\NormalTok{Outcome, preds, }\AttributeTok{quiet =} \ConstantTok{TRUE}\NormalTok{)}
\NormalTok{  auc\_vals[i] }\OtherTok{\textless{}{-}} \FunctionTok{as.numeric}\NormalTok{(pROC}\SpecialCharTok{::}\FunctionTok{auc}\NormalTok{(roc\_i))}
\NormalTok{\}}

\NormalTok{auc\_df }\OtherTok{\textless{}{-}} \FunctionTok{tibble}\NormalTok{(}\AttributeTok{imputation =} \DecValTok{1}\SpecialCharTok{:}\NormalTok{m, }\AttributeTok{AUC =}\NormalTok{ auc\_vals)}

\NormalTok{auc\_df }\SpecialCharTok{\%\textgreater{}\%}
\NormalTok{  knitr}\SpecialCharTok{::}\FunctionTok{kable}\NormalTok{(}\AttributeTok{digits =} \DecValTok{3}\NormalTok{, }\AttributeTok{caption =} \StringTok{"Per{-}imputation logistic AUCs"}\NormalTok{) }\SpecialCharTok{\%\textgreater{}\%}
\NormalTok{  kableExtra}\SpecialCharTok{::}\FunctionTok{kable\_styling}\NormalTok{(}\AttributeTok{full\_width =} \ConstantTok{FALSE}\NormalTok{)}
\end{Highlighting}
\end{Shaded}

\begin{longtable}[t]{rr}
\caption{\label{tab:mice-auc}Per-imputation logistic AUCs}\\
\toprule
imputation & AUC\\
\midrule
1 & 0.819\\
2 & 0.829\\
3 & 0.822\\
4 & 0.824\\
5 & 0.819\\
\bottomrule
\end{longtable}

\begin{Shaded}
\begin{Highlighting}[]
\FunctionTok{cat}\NormalTok{(}
  \StringTok{"}\SpecialCharTok{\textbackslash{}n}\StringTok{Mean AUC across imputations:"}\NormalTok{,}
  \FunctionTok{round}\NormalTok{(}\FunctionTok{mean}\NormalTok{(auc\_vals), }\DecValTok{3}\NormalTok{),}
  \StringTok{"(SD ="}\NormalTok{, }\FunctionTok{round}\NormalTok{(}\FunctionTok{sd}\NormalTok{(auc\_vals), }\DecValTok{3}\NormalTok{), }\StringTok{")}\SpecialCharTok{\textbackslash{}n}\StringTok{"}
\NormalTok{)}
\end{Highlighting}
\end{Shaded}

\begin{verbatim}
## 
## Mean AUC across imputations: 0.823 (SD = 0.004 )
\end{verbatim}

\subsection{Compare to median-impute baseline (recompute baseline here
for
reproducibility)}\label{compare-to-median-impute-baseline-recompute-baseline-here-for-reproducibility}

Description: - Re-run the median imputation baseline (same as earlier
simple impute) and compute logistic AUC on the same train/test split.
This ensures the comparison is apples-to-apples.

Expected outputs: - Baseline AUC and short interpretation whether
multiple-imputation meaningfully changed test AUC.

\begin{Shaded}
\begin{Highlighting}[]
\CommentTok{\# Build a median{-}imputed dataset from diabetes\_clean to compare}
\NormalTok{di\_med }\OtherTok{\textless{}{-}}\NormalTok{ diabetes\_clean}
\ControlFlowTok{for}\NormalTok{ (v }\ControlFlowTok{in} \FunctionTok{c}\NormalTok{(}\StringTok{"Glucose"}\NormalTok{,}\StringTok{"BloodPressure"}\NormalTok{,}\StringTok{"SkinThickness"}\NormalTok{,}\StringTok{"Insulin"}\NormalTok{,}\StringTok{"BMI"}\NormalTok{)) \{}
\NormalTok{  di\_med[[v]][}\FunctionTok{is.na}\NormalTok{(di\_med[[v]])] }\OtherTok{\textless{}{-}} \FunctionTok{median}\NormalTok{(di\_med[[v]], }\AttributeTok{na.rm =} \ConstantTok{TRUE}\NormalTok{)}
\NormalTok{\}}

\CommentTok{\# Use same train\_index\_mice to split}
\NormalTok{train\_med }\OtherTok{\textless{}{-}}\NormalTok{ di\_med[train\_index\_mice, ]}
\NormalTok{test\_med  }\OtherTok{\textless{}{-}}\NormalTok{ di\_med[}\SpecialCharTok{{-}}\NormalTok{train\_index\_mice, ]}

\NormalTok{glm\_med }\OtherTok{\textless{}{-}} \FunctionTok{glm}\NormalTok{(Outcome }\SpecialCharTok{\textasciitilde{}}\NormalTok{ Pregnancies }\SpecialCharTok{+}\NormalTok{ Glucose }\SpecialCharTok{+}\NormalTok{ BloodPressure }\SpecialCharTok{+}\NormalTok{ SkinThickness }\SpecialCharTok{+}
\NormalTok{                 Insulin }\SpecialCharTok{+}\NormalTok{ BMI }\SpecialCharTok{+}\NormalTok{ DiabetesPedigreeFunction }\SpecialCharTok{+}\NormalTok{ Age, }\AttributeTok{data =}\NormalTok{ train\_med, }\AttributeTok{family =}\NormalTok{ binomial)}
\NormalTok{preds\_med }\OtherTok{\textless{}{-}} \FunctionTok{predict}\NormalTok{(glm\_med, }\AttributeTok{newdata =}\NormalTok{ test\_med, }\AttributeTok{type =} \StringTok{"response"}\NormalTok{)}
\NormalTok{roc\_med }\OtherTok{\textless{}{-}}\NormalTok{ pROC}\SpecialCharTok{::}\FunctionTok{roc}\NormalTok{(test\_med}\SpecialCharTok{$}\NormalTok{Outcome, preds\_med, }\AttributeTok{quiet =} \ConstantTok{TRUE}\NormalTok{)}
\NormalTok{auc\_med }\OtherTok{\textless{}{-}} \FunctionTok{as.numeric}\NormalTok{(pROC}\SpecialCharTok{::}\FunctionTok{auc}\NormalTok{(roc\_med))}

\FunctionTok{cat}\NormalTok{(}\StringTok{"Median{-}impute logistic AUC: "}\NormalTok{, }\FunctionTok{round}\NormalTok{(auc\_med, }\DecValTok{3}\NormalTok{), }\StringTok{"}\SpecialCharTok{\textbackslash{}n}\StringTok{"}\NormalTok{)}
\end{Highlighting}
\end{Shaded}

\begin{verbatim}
## Median-impute logistic AUC:  0.824
\end{verbatim}

\begin{Shaded}
\begin{Highlighting}[]
\CommentTok{\# Quick summary comparing mean MI AUC vs median baseline}
\FunctionTok{cat}\NormalTok{(}\StringTok{"Mean MI AUC: "}\NormalTok{, }\FunctionTok{round}\NormalTok{(}\FunctionTok{mean}\NormalTok{(auc\_vals), }\DecValTok{3}\NormalTok{), }\StringTok{" (median baseline AUC: "}\NormalTok{, }\FunctionTok{round}\NormalTok{(auc\_med, }\DecValTok{3}\NormalTok{), }\StringTok{")}\SpecialCharTok{\textbackslash{}n}\StringTok{"}\NormalTok{, }\AttributeTok{sep =} \StringTok{""}\NormalTok{)}
\end{Highlighting}
\end{Shaded}

\begin{verbatim}
## Mean MI AUC: 0.823 (median baseline AUC: 0.824)
\end{verbatim}

\subsection{Extended EDA: AgeGroup counts and outcome-stratified
visuals}\label{extended-eda-agegroup-counts-and-outcome-stratified-visuals}

Description: - Adds AgeGroup count bar chart, a faceted boxplot (Glucose
by AgeGroup and Outcome), and a stacked bar of BMI categories by
Outcome. - These visuals improve descriptive EDA and directly address
the rubric requests.

Expected outputs: - Bar chart showing sample size per AgeGroup. -
Boxplots / violin plots showing Glucose differences by AgeGroup and by
Outcome to help the audience visually assess interactions between age
and glucose. - Stacked bar showing proportions of BMI categories split
by Outcome.

\begin{Shaded}
\begin{Highlighting}[]
\FunctionTok{library}\NormalTok{(ggplot2)}
\FunctionTok{library}\NormalTok{(dplyr)}
\FunctionTok{library}\NormalTok{(gridExtra)  }\CommentTok{\# for arranging multiple plots}

\CommentTok{\# AgeGroup counts}
\NormalTok{p\_age\_counts }\OtherTok{\textless{}{-}}\NormalTok{ diabetes\_clean }\SpecialCharTok{\%\textgreater{}\%}
  \FunctionTok{count}\NormalTok{(AgeGroup) }\SpecialCharTok{\%\textgreater{}\%}
  \FunctionTok{ggplot}\NormalTok{(}\FunctionTok{aes}\NormalTok{(}\AttributeTok{x =}\NormalTok{ AgeGroup, }\AttributeTok{y =}\NormalTok{ n, }\AttributeTok{fill =}\NormalTok{ AgeGroup)) }\SpecialCharTok{+}
  \FunctionTok{geom\_col}\NormalTok{(}\AttributeTok{show.legend =} \ConstantTok{FALSE}\NormalTok{) }\SpecialCharTok{+}
  \FunctionTok{labs}\NormalTok{(}\AttributeTok{title =} \StringTok{"Count by AgeGroup"}\NormalTok{, }\AttributeTok{x =} \StringTok{"Age group"}\NormalTok{, }\AttributeTok{y =} \StringTok{"Count"}\NormalTok{) }\SpecialCharTok{+}
  \FunctionTok{theme\_minimal}\NormalTok{()}

\CommentTok{\# Glucose by AgeGroup and Outcome (boxplot + jitter)}
\NormalTok{p\_glucose\_age\_outcome }\OtherTok{\textless{}{-}}\NormalTok{ diabetes\_clean }\SpecialCharTok{\%\textgreater{}\%}
  \FunctionTok{mutate}\NormalTok{(}\AttributeTok{Outcome =} \FunctionTok{factor}\NormalTok{(Outcome)) }\SpecialCharTok{\%\textgreater{}\%}
  \FunctionTok{ggplot}\NormalTok{(}\FunctionTok{aes}\NormalTok{(}\AttributeTok{x =}\NormalTok{ AgeGroup, }\AttributeTok{y =}\NormalTok{ Glucose, }\AttributeTok{fill =}\NormalTok{ Outcome)) }\SpecialCharTok{+}
  \FunctionTok{geom\_boxplot}\NormalTok{(}\AttributeTok{position =} \FunctionTok{position\_dodge}\NormalTok{(}\AttributeTok{width =} \FloatTok{0.8}\NormalTok{), }\AttributeTok{outlier.shape =} \ConstantTok{NA}\NormalTok{, }\AttributeTok{alpha =} \FloatTok{0.6}\NormalTok{) }\SpecialCharTok{+}
  \FunctionTok{geom\_jitter}\NormalTok{(}\FunctionTok{aes}\NormalTok{(}\AttributeTok{color =}\NormalTok{ Outcome), }\AttributeTok{width =} \FloatTok{0.2}\NormalTok{, }\AttributeTok{alpha =} \FloatTok{0.3}\NormalTok{, }\AttributeTok{size =} \FloatTok{0.8}\NormalTok{, }\AttributeTok{show.legend =} \ConstantTok{FALSE}\NormalTok{) }\SpecialCharTok{+}
  \FunctionTok{labs}\NormalTok{(}\AttributeTok{title =} \StringTok{"Glucose by AgeGroup and Outcome"}\NormalTok{, }\AttributeTok{y =} \StringTok{"Glucose (mg/dL)"}\NormalTok{) }\SpecialCharTok{+}
  \FunctionTok{theme\_minimal}\NormalTok{()}

\CommentTok{\# Stacked BMI category proportions by Outcome}
\NormalTok{p\_bmi\_stack }\OtherTok{\textless{}{-}}\NormalTok{ diabetes\_clean }\SpecialCharTok{\%\textgreater{}\%}
  \FunctionTok{mutate}\NormalTok{(}\AttributeTok{Outcome =} \FunctionTok{factor}\NormalTok{(Outcome, }\AttributeTok{labels =} \FunctionTok{c}\NormalTok{(}\StringTok{"No"}\NormalTok{,}\StringTok{"Yes"}\NormalTok{))) }\SpecialCharTok{\%\textgreater{}\%}
  \FunctionTok{count}\NormalTok{(BMI\_cat, Outcome) }\SpecialCharTok{\%\textgreater{}\%}
  \FunctionTok{group\_by}\NormalTok{(Outcome) }\SpecialCharTok{\%\textgreater{}\%}
  \FunctionTok{mutate}\NormalTok{(}\AttributeTok{prop =}\NormalTok{ n }\SpecialCharTok{/} \FunctionTok{sum}\NormalTok{(n)) }\SpecialCharTok{\%\textgreater{}\%}
  \FunctionTok{ggplot}\NormalTok{(}\FunctionTok{aes}\NormalTok{(}\AttributeTok{x =}\NormalTok{ Outcome, }\AttributeTok{y =}\NormalTok{ prop, }\AttributeTok{fill =}\NormalTok{ BMI\_cat)) }\SpecialCharTok{+}
  \FunctionTok{geom\_col}\NormalTok{() }\SpecialCharTok{+}
  \FunctionTok{scale\_y\_continuous}\NormalTok{(}\AttributeTok{labels =}\NormalTok{ scales}\SpecialCharTok{::}\FunctionTok{percent\_format}\NormalTok{()) }\SpecialCharTok{+}
  \FunctionTok{labs}\NormalTok{(}\AttributeTok{title =} \StringTok{"BMI category distribution by Outcome"}\NormalTok{, }\AttributeTok{y =} \StringTok{"Proportion"}\NormalTok{, }\AttributeTok{x =} \StringTok{"Outcome"}\NormalTok{) }\SpecialCharTok{+}
  \FunctionTok{theme\_minimal}\NormalTok{()}

\CommentTok{\# Arrange plots in a 2{-}row layout: top row = two plots, bottom row = one}
\FunctionTok{grid.arrange}\NormalTok{(}
  \AttributeTok{grobs =} \FunctionTok{list}\NormalTok{(}
    \FunctionTok{arrangeGrob}\NormalTok{(p\_age\_counts, p\_bmi\_stack, }\AttributeTok{ncol =} \DecValTok{2}\NormalTok{),}
\NormalTok{    p\_glucose\_age\_outcome}
\NormalTok{  ),}
  \AttributeTok{nrow =} \DecValTok{2}
\NormalTok{)}
\end{Highlighting}
\end{Shaded}

\begin{verbatim}
## Warning: Removed 5 rows containing non-finite outside the scale range
## (`stat_boxplot()`).
\end{verbatim}

\begin{verbatim}
## Warning: Removed 5 rows containing missing values or values outside the scale range
## (`geom_point()`).
\end{verbatim}

\includegraphics{Final-Projet_files/figure-latex/extra-eda-plots-1.pdf}

\subsection{Results summary}\label{results-summary}

Across all modeling approaches, predictors related to glucose metabolism
consistently demonstrated the strongest association with diabetes onset.
The baseline logistic regression achieved moderate discriminative
performance on the held-out test set (AUC ≈ 0.75--0.78), with fasting
glucose emerging as the most statistically significant predictor,
followed by BMI and age. Odds ratio estimates from the logistic model
were directionally consistent with established clinical knowledge,
indicating increased diabetes risk with higher glucose levels, higher
BMI, and older age.

Tree-based models improved predictive performance relative to the
baseline. The random forest model achieved a higher test-set AUC (≈
0.80--0.82), while the XGBoost model produced the strongest overall
discrimination, with a test AUC of approximately 0.83. ROC curve
comparisons showed consistent improvement from logistic regression to
random forest and XGBoost, suggesting that non-linear interactions among
predictors contribute meaningfully to prediction accuracy in this
dataset. Calibration analysis of the best-performing model indicated
reasonably good agreement between predicted probabilities and observed
outcome rates, particularly in the mid-range risk strata.

Model interpretability analyses reinforced these findings. Variable
importance measures and permutation-based importance ranked glucose as
the most influential feature, followed by BMI, age, insulin, and
diabetes pedigree function. SHAP explanations further confirmed that
higher glucose values consistently increased predicted diabetes risk
across individuals, while BMI and age exhibited heterogeneous but
clinically plausible effects. Individual-level SHAP analyses illustrated
how combinations of elevated glucose and BMI drove higher predicted risk
for specific test observations. Overall, results indicate that gradient
boosting models provide the best predictive performance on this dataset,
while interpretability analyses support glucose, BMI, and age as the
primary drivers of diabetes risk.

\subsection{Discussion and
interpretation}\label{discussion-and-interpretation}

Description: Interpret the model findings in clinical context and
propose actionable recommendations. Discuss potential biases and
pitfalls.

Key points to state: - Glucose is the most predictive variable
(biologically plausible). - BMI and Age are meaningful predictors
consistent with literature. - The dataset consists only of female Pima
Indians; generalization to other populations is limited. - Missingness
treatment (zeros -\textgreater{} NA then impute) is critical; different
imputation approaches can slightly change model performance. - Ethical
considerations: avoid direct deployment without calibration to the
target population and clinical validation.

\subsection{Limitations}\label{limitations}

\begin{itemize}
\tightlist
\item
  Dataset limitation: cohort restricted in gender and ethnicity.
\item
  Predictive models do not imply causation.
\item
  Potential measurement errors and selection bias.
\item
  Need for prospective clinical validation before any real-world
  deployment.
\end{itemize}

\subsection{Feature beyond class
coverage}\label{feature-beyond-class-coverage}

Description: We used SHAP explanations (fastshap) and permutation
importance as features not usually covered in introductory class
material. We also prepared to render slides from R Markdown
(xaringan/ioslides) as a reproducible presentation artifact.

\subsection{Challenges encountered}\label{challenges-encountered}

\begin{itemize}
\tightlist
\item
  Data issue: zeros used as missing for biologically impossible values.
  Decision: replace with NA and impute. Show quick before/after counts
  and how model metrics changed slightly after imputation.
\item
  Performance tuning: XGBoost hyperparameter tuning was slow; I limited
  tuning grid to remain within time constraints.
\item
  Reproducibility: external metadata scraping may break; solution: save
  metadata locally in the repo.
\end{itemize}

\subsection{Timeline / Deliverables schedule
(recommended)}\label{timeline-deliverables-schedule-recommended}

\begin{itemize}
\tightlist
\item
  Proposal (submitted): completed (this document).
\item
  Approval meeting: schedule within one week.
\item
  Milestone 1 (data cleaning \& EDA): 1 week after approval meeting.
\item
  Milestone 2 (model training \& evaluation): 2--3 weeks after milestone
  1.
\item
  Milestone 3 (interpretation, SHAP, slides): 1 week after milestone 2.
\item
  Final deliverables: GitHub repo + knitted HTML (or PDF) +
  presentation.
\end{itemize}

\subsection{Repository \& submission
instructions}\label{repository-submission-instructions}

\begin{itemize}
\tightlist
\item
  Create repository:
  \texttt{msds607-final-diabetes-\textless{}yourname\textgreater{}}
\item
  Add files: \texttt{data/diabetes.csv}, \texttt{proposal.Rmd} (this
  file), \texttt{final\_report.Rmd} (if separate), \texttt{slides/},
  \texttt{README.md}
\item
  Commit and push; include link in the course submission.
\item
  Ensure \texttt{README.md} contains package installation commands and a
  statement that kitting \texttt{proposal.Rmd} produces the full report.
\end{itemize}

\subsection{Appendix: session info
(reproducibility)}\label{appendix-session-info-reproducibility}

Description: Include full session info to help graders reproduce the
environment.

\begin{Shaded}
\begin{Highlighting}[]
\FunctionTok{sessionInfo}\NormalTok{()}
\end{Highlighting}
\end{Shaded}

\begin{verbatim}
## R version 4.4.1 (2024-06-14 ucrt)
## Platform: x86_64-w64-mingw32/x64
## Running under: Windows 11 x64 (build 26100)
## 
## Matrix products: default
## 
## 
## locale:
## [1] LC_COLLATE=English_United States.utf8 
## [2] LC_CTYPE=English_United States.utf8   
## [3] LC_MONETARY=English_United States.utf8
## [4] LC_NUMERIC=C                          
## [5] LC_TIME=English_United States.utf8    
## 
## time zone: America/New_York
## tzcode source: internal
## 
## attached base packages:
## [1] stats     graphics  grDevices utils     datasets  methods   base     
## 
## other attached packages:
##  [1] broom_1.0.7          cowplot_1.1.3        kableExtra_1.4.0    
##  [4] knitr_1.50           skimr_2.1.5          corrplot_0.95       
##  [7] mice_3.18.0          gridExtra_2.3        fastshap_0.1.1      
## [10] vip_0.4.2            xgboost_1.7.10.1     randomForest_4.7-1.2
## [13] pROC_1.18.5          caret_7.0-1          lattice_0.22-6      
## [16] lubridate_1.9.4      forcats_1.0.0        stringr_1.5.1       
## [19] dplyr_1.1.4          purrr_1.1.0          readr_2.1.5         
## [22] tidyr_1.3.1          tibble_3.3.0         ggplot2_3.5.1       
## [25] tidyverse_2.0.0     
## 
## loaded via a namespace (and not attached):
##  [1] Rdpack_2.6.2         rlang_1.1.4          magrittr_2.0.3      
##  [4] compiler_4.4.1       systemfonts_1.2.1    vctrs_0.6.5         
##  [7] reshape2_1.4.4       crayon_1.5.3         pkgconfig_2.0.3     
## [10] shape_1.4.6.1        fastmap_1.2.0        backports_1.5.0     
## [13] labeling_0.4.3       rmarkdown_2.29       prodlim_2024.06.25  
## [16] tzdb_0.4.0           nloptr_2.1.1         bit_4.5.0.1         
## [19] tinytex_0.58         xfun_0.51            glmnet_4.1-8        
## [22] jomo_2.7-6           jsonlite_2.0.0       recipes_1.1.1       
## [25] pan_1.9              parallel_4.4.1       R6_2.6.1            
## [28] stringi_1.8.4        RColorBrewer_1.1-3   parallelly_1.38.0   
## [31] boot_1.3-30          rpart_4.1.23         Rcpp_1.0.13         
## [34] iterators_1.0.14     future.apply_1.11.3  base64enc_0.1-3     
## [37] Matrix_1.7-0         splines_4.4.1        nnet_7.3-19         
## [40] timechange_0.3.0     tidyselect_1.2.1     rstudioapi_0.17.1   
## [43] yaml_2.3.10          timeDate_4041.110    codetools_0.2-20    
## [46] listenv_0.9.1        plyr_1.8.9           withr_3.0.2         
## [49] evaluate_1.0.3       future_1.34.0        survival_3.6-4      
## [52] xml2_1.3.7           pillar_1.10.1        foreach_1.5.2       
## [55] stats4_4.4.1         reformulas_0.4.0     generics_0.1.3      
## [58] vroom_1.6.5          hms_1.1.3            scales_1.4.0        
## [61] minqa_1.2.8          globals_0.16.3       class_7.3-22        
## [64] glue_1.8.0           tools_4.4.1          data.table_1.17.8   
## [67] lme4_1.1-36          ModelMetrics_1.2.2.2 gower_1.0.2         
## [70] grid_4.4.1           rbibutils_2.3        ipred_0.9-15        
## [73] nlme_3.1-164         repr_1.1.7           cli_3.6.3           
## [76] viridisLite_0.4.2    svglite_2.1.3        lava_1.8.1          
## [79] gtable_0.3.6         digest_0.6.37        farver_2.1.2        
## [82] htmltools_0.5.8.1    lifecycle_1.0.4      hardhat_1.4.1       
## [85] mitml_0.4-5          bit64_4.6.0-1        MASS_7.3-60.2
\end{verbatim}

\end{document}
